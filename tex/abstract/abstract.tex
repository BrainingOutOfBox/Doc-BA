\thispagestyle{empty}
\section*{Abstract}
%1. Kontext
%2. Problem
%3. Lösung
%4. Warum gute Lösung
%5. Ausblick

Die papiergestützte Methode 635 ist eine Kreativitäts- und Brainwriting-Technik, welche die Entwicklung von neuen, ungewöhnlichen Ideen für Problemlösungen in der Gruppe fördert. Nach unserem Wissensstand existiert für die Methode 635 noch keine mobile Applikation. 

Als Resultat der vorliegenden Bachelorarbeit ist eine stabile und performante Cross-Plattform Applikation für iOS und Android hervorgegangen. Diese ermöglicht es Benutzern die Methode 635 auf ihrem Smartphone oder Tablet anzuwenden. Zusätzlich zur Eingabe von Ideen via Text ist es möglich Skizzen als Lösungsvorschlag zu zeichnen oder aus einer Liste von vordefinierten Patterns ein passendes Pattern auszuwählen. Auch ist es möglich das erarbeitete Brainstorming als Markdown zu exportieren. Als Teil der Dokumentation beschreiben wir zudem die Herausforderungen, die während der Umsetzung aufgetreten sind. So können unsere Erfahrungen anderen Entwicklern bei ähnlichen Projekten helfen oder einzelne Risiken gar komplett verringern.

Die Möglichkeit Skizzen zu zeichnen und Patterns aus der vordefinierten Patternsprache (MAP) zu verwenden, kommt gerade (aber nicht ausschliesslich) Software-Architekten zugute. So können sie diese Funktionen im Zuge der kreativen (Design-)Entwurfsphase nutzen, um kollaborativ an einer allfälligen Lösung zu arbeiten.

Als Ausblick für die vorliegende Arbeit wäre es denkbar, weitere Ideen-Typen zu integrieren oder den Fokus etwas zu weiten und andere Brainstorming-Methoden oder Konzepte wie z.B. die World-Cafe Methode zu integrieren. 


%
%
%Die vorliegende Bachelorarbeit befasst sich mit der Software Engineering Aktivität 'Solution Strategy', welche ein hohes Mass an kollaborativen Problemlösungsprozessen erfordert. Die Methode 635 ist eine Kreativitäts- und Brainwriting-Technik, welche die Entwicklung von neuen, ungewöhnlichen Ideen für Problemlösungen in der Gruppe fördert. Sie dient als praktische Umsetzung der Solution Strategy.
%
%Wie schon in der Studienarbeit, führten wir auch hier zunächst eine Vorstudie durch. Wir beschäftigen uns dabei mit der Frage, was die Solution Strategy genau ist und wie sich diese in unser bestehendes Projekt integrieren lässt. Ebenfalls musste geklärt werden wie Bilder in die bereits verwendete MongoDB-Datenbank gespeichert werden können, da dies für einzelne geplante Funktionen von essenzieller Bedeutung war. Des Weitern mussten wir uns ein Konzept überlegen, wie es möglich sein soll, die bestehende Applikation um weitere Ideen-Typen zu erweitern, sodass wir diese Überlegungen im Zuge des Refactorings auch gleich berücksichtigen konnten. 
%
%Aus diesen und weiteren Analysen entwickelten wir einzelne Prototypen, um die Machbarkeit der geplanten Funktionen festzustellen. Da sich diese wie angedacht umsetzen liessen, konnten die Funktionen in die bestehende Applikation integriert werden. 
%
%Als Resultat der Bachelorarbeit ist eine stabile und performante Cross-Plattform Applikation für iOS und Android hervorgegangen. Diese ermöglicht es Benutzern die Methode 635 auf ihrem Smartphone oder Tablet anzuwenden. Anders als bei der Studienarbeit ist es nun auch möglich Skizzen als Lösungsvorschlag zu zeichnen, aus einer Liste von vordefinierten Pattern ein passendes Pattern auszuwählen sowie das erarbeitete Brainstorming als Markdown zu exportieren. Als Teil der Dokumentation beschreiben wir auch die Herausforderungen, die während der Umsetzung aufgetreten sind. So können unsere Erfahrungen anderen Entwicklern bei ähnlichen Projekten helfen oder einzelne Risiken gar komplett verringern.
%
%Als Ausblick für die vorliegende Arbeit wäre es denkbar, weitere Ideen-Typen zu integrieren oder den Fokus etwas zu weiten und andere Brainstorming-Methoden oder Konzepte wie z.B. die World-Cafe Methode zu integrieren.   