\thispagestyle{empty}
\section*{Abstract}
Die vorliegende Bachelorarbeit befasst sich mit der Software Engineering Aktivität 'Solution Strategy', welche ein hohes Mass an kollaborativen Problemlösungsprozessen erfordert. Herr Gernot Starke beschreibt darin, wie man in der Software Architektur die Ansätze zur Lösungsfindung erarbeiten und dokumentieren kann. Die Methode 635 ist eine Kreativitäts- und Brainwriting-Technik, welche die Entwicklung von neuen, ungewöhnlichen Ideen für Problemlösungen in der Gruppe fördert. Sie dient daher als praktische Umsetzung der Solution Strategy.

Es soll an dieser Stelle noch erwähnt werden, dass dieser Bachelorarbeit eine Studienarbeit \cite{methode635-sa} vorher ging.

Wie schon in der Studienarbeit, führten wir auch hier zunächst eine Vorstudie durch. Wir beschäftigen uns dabei mit der Frage, was die Solution Strategy genau ist und wie sich diese in unser bestehendes Projekt integrieren lässt. Ebenfalls musste geklärt werden wie Bilder in die bereits verwendete MongoDB-Datenbank gespeichert werden können, da dies für einzelne geplante Funktionen von essenzieller Bedeutung war. Des Weitern mussten wir uns ein Konzept überlegen, wie es möglich sein soll, die bestehende Applikation um weitere Ideen-Typen zu erweitern, sodass wir diese Überlegungen im Zuge des Refactorings auch gleich berücksichtigen konnten. 

Aus diesen und weiteren Analysen entwickelten wir einzelne Prototypen, um die Machbarkeit der geplanten Funktionen festzustellen. Da sich diese wie angedacht umsetzen liessen, konnten die Funktionen in die bestehende Applikation integriert werden. 

Als Resultat der Bachelorarbeit ist eine lauffähige Cross-Plattform Applikation für iOS und Android hervorgegangen. Diese ermöglicht es Benutzern die Methode 635 auf ihrem Smartphone oder Tablet anzuwenden. Im Unterschied zur vorangegangener Studienarbeit konnte die Applikation einem kompletten Refactoring unterzogen werden, was zu einer erheblichen Verbesserung der Performance und Stabilität führte. Ausserdem ist es nun möglich Skizzen als Lösungsvorschlag zu zeichnen, aus einer Liste von Pattern ein passendes Pattern auszuwählen sowie das erarbeitete Brainstorming als Markdown zu exportieren. Als Teil der Dokumentation beschreiben wir auch die Herausforderungen, die während der Umsetzung aufgetreten sind. So können unsere Erfahrungen anderen Entwicklern bei ähnlichen Projekten helfen oder einzelne Risiken gar komplett verringern.   