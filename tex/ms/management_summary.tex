\section{Management Summary}

\subsection{Ausgangslage}
Die papiergestützte Methode 635 ist eine Kreativitäts- und Brainwriting-Technik, für die es bisher noch keine Unterstützung in Form von mobilen Apps gab. Eine Studienarbeit im Herbstsemester 2018/2019 konzipierte und implementierte daher einen ersten Prototypen einer Smartphone App für die Methode 635. Durch die Verwendung von Xamarin konnte sowohl die Unterstützung für iOS wie auch für Android Smartphones ermöglicht werden.

\subsection{Vorgehen}
Für diese Arbeit lässt sich unser Vorgehen in drei Phasen einteilen. In der ersten Phase fokussierten wir uns primär auf Recherchearbeiten und konzeptionelle Arbeiten. Dabei führten wir eine Vorstudie durch und beschäftigten uns mit der Frage, was die Solution Strategy ist und wie sich diese in das bestehende Projekt einbetten lässt. Weiter überlegten wir uns, wie Bilder in die bereits verwendete dokumentorientierte Datenbank abgelegt werden können. Da Software-Architekten häufig Skizzen für z.B. Designentwürfe anfertigen, stellte dies ein grundlegender Bestandteil der geplanten Funktionen dar. Um die bestehende Applikation für weitere Ideen-Typen zu erweitern, investierten wir einige Zeit, ein entsprechendes Konzept dafür zu entwickeln.

In der zweiten Phase unterzogen wir sowohl das Backend wie auch die Cross-Plattform App einem umfangreichen Refactoring. Dies sollte uns helfen den bestehenden Code, welcher in der vorhergegangenen Studienarbeit geschrieben wurde, zu vereinfachen und dahingehend anzupassen, um ihn für die geplanten Funktionen verwenden zu können.

In der dritten und letzten Phase ging es nun darum, die geplanten Funktionen und Erweiterungen in Form von Code zu implementieren. Um die Machbarkeit jener Funktionen festzustellen, wurden vereinzelt Prototypen dafür geschrieben. Da sich diese wie gewünscht umsetzen liessen, konnten die Funktionen in die bestehende Applikation integriert werden. 

Um sicherzustellen, dass das gesamte System wie gewünscht funktionierte, führten wir, zusätzlich zum manuellen Testen während der Entwicklung, gegen Ende der Arbeit mehrere User-Tests durch.

\subsection{Ergebnisse}
Aus der vorliegenden Bachelorarbeit ist eine stabile und performante Cross-Plattform Applikation für Android und iOS hervorgegangen. Diese ermöglicht es Benutzern die Methode 635 auf dem eigenen Smartphone oder Tablet durchzuführen.

Die in der Vorstudie gemachten Überlegungen in Bezug auf die Solution Strategy verhalfen uns zu einem Konzept (siehe Tabelle \ref{tab:arc42-mapping}), welches die Gestaltung des Lösungs\-findungs\-prozess in unserer Applikation beschreibt.  

Zusätzlich zu den bestehenden Funktionen aus der Studienarbeit schufen wir mit dieser Arbeit die Möglichkeit, Skizzen als Lösungsvorschlag zu zeichnen, aus einer Liste von Pattern ein passendes Pattern auszuwählen sowie das erarbeitete Brainstorming als Markdown zu exportieren. Die Möglichkeit Skizzen zu zeichnen und Pattern aus der vordefinierten Patternsprache (MAP) zu verwenden, kommt gerade (aber nicht ausschliesslich) Software-Architekten zu Gute. So können sie diese Funktionen im Zuge der kreativen (Design-)Entwurfsphase nutzen, um kollaborativ an einer allfälligen Lösung zu arbeiten.

Dank des Code-Refactorings zu Beginn der Bachelorarbeit konnte sowohl die Performance wie auch die Stabilität deutlich verbessert werden. Aus Rückmeldungen von User-Tests konnten ausserdem Verbesserungen hinsichtlich Userführung und UI-Design durchgeführt werden.