
\thispagestyle{empty}
\section*{Aufgabenstellung}
Die komplette Aufgabenstellung wurde von Prof. Dr. Olaf Zimmermann erstellt und kann dem Anhang \ref{sec:aufgabenstellung} entnommen werden.

\subsection*{Ausgangslage}
Die papiergestützte Methode 635 ist eine Kreativitäts- und Brainwriting-Technik, für die es bisher noch keine Unterstützung in mobilen Apps gab. Eine Studienarbeit im Herbstsemester 2018/2019 konzipierte und implementierte daher einen ersten Prototyp einer SmartPhone App für die Methode 635; Cross Platform Support (Android- und iOS) wurde durch Verwendung von Xamarin erreicht. Ein Vorteil einer derartigen mobilen Anwendung ist, dass Anwender die Methode 635 nutzen können, wenn sie sich nicht ständig an einem Ort befinden. 


In dieser Bachelorarbeit sollen weitere Features implementiert werden auf Basis der Testergebnisse für den bestehenden Prototypen. Im Fokus steht dabei insbesondere die Software Engineering Aktivität "Solution Strategy" aus dem Buch „Effektive Softwarearchitekturen“ von G. Starke, die in der HSR-Vorlesung Application Architecture behandelt wird. Diese Aktivität erfordert ein hohes Mass an kollaborativen Problemlösungsprozessen auf unterschiedlichen Abstraktionsebenen (z.B. Konzepte, Technologien, und Produkte). 

\subsection*{Ziele der Arbeit und Liefergegenstände}\label{subsec:ziele}
Die Vision der Arbeit ist es, die Papierversion für diese Methodik zu funktional und qualitativ zu überbieten. Damit die erweiterte App einen Mehrwert gegenüber der Papierversion bietet, soll es z.B. möglich sein, verschiedene Medien (Text, Video, Bilder, etc.) zu verwenden bzw. einzubinden. Hardware-Features und andere Apps sollen – wenn sinnvoll und technisch möglich – integriert werden.

Bei der Umsetzung spielen Erfolgsfaktoren wie einfache und intuitive Bedienung der App und ein unkompliziertes Reporting sowie Robustheit und Stabilität (Bsp. keine Zeit- und Datenverluste) eine wichtige Rolle. Ein wichtiger Input für die Bachelorarbeit ist das externe Feedback zu den Ergebnissen der vorangegangenen Studienarbeit, zum Beispiel Feedback zur Usability und zur Codequalität.

Weitere kritische Erfolgsfaktoren sind:
\begin{itemize}
  \item Konfigurierbarkeit und Erweiterbarkeit (im Hinblick auf Folgearbeiten, die u.U. auch andere Brainstorming- und Innovationsmethoden wie z.B. Design Thinking unterstützen).
  \item Sinnvolle Ausnutzung der Smartphone-Fähigkeiten, um einen Mehrwert im Vergleich zur traditionellen, papiergestützten Methode zu erreichen. 
  \item Validierung der Konzepte und ihrer Implementierung mit Hilfe von User Tests in mindestens einem Anwendungsbereich (Bsp. Architekturentscheidungen und -optionen).
\end{itemize}

