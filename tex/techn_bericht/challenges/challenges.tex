\subsection{Herausforderungen}
Hier sind besonders erwähnenswerte Herausforderungen und Hürden beschrieben, die im Verlaufe des Projektes aufkamen. Dies soll anderen Software Ingenieuren oder Interessierten helfen, aus unseren Schwierigkeiten zu lernen. 


\subsubsection{Navigationsprobleme mit Prism Forms}\label{subsub:navigation}

Die grösste Herausforderung im Frontend war die Navigation mit Prism Forms und Xamarin Forms. Ein zentraler Unterschied liegt in der absolution und relativen Navigation. Bei der absoluten Navigation wird der gesamte Navigationsstack bei jedem Aufruf neu aufgebaut, was zur Folge hat, dass alle beteiligten ViewModels neu erstellt (Konstruktoraufruf) werden und darauf navigiert wird (Aufruf von \texttt{OnNavigatedTo} und \texttt{OnNavigatedFrom}). Dies im Unterschied zur relativen Navigation, wo jeweils die zu navigierende Page auf den Stack gelegt wird und nur die Methoden auf dem zuständigen ViewModel aufgerufen werden. 

Das Problem war nun, dass wir uns im Kontext einer \texttt{TabbedPage} bewegen und die Navigationsschritte bei einem Klick auf einem Element in einem Tab auf ein anderes Tab wechseln müssen. Leider funktioniert in diesem Zusammenhang nur die absolute Navigation, weil wir innerhalb einer Page mit den verschiedenen Tabs navigieren wollen und die Navigation von Prism Forms dazu implementiert wurde, zwischen Pages zu navigieren. Dies hat zur Folge, dass jeder Tabwechsel mit absoluter Navigation entwickelt werden musste. Der gesamte Code muss also berücksichtigen, dass die ViewModels mehrmals initialisiert werden und somit Instanzen überschrieben und verändert werden können. Dies ist ein erheblicher Mehraufwand, der immer wieder zu Fehlverhalten und viel zusätzlichem Code führte. 

Im späteren Projektverlauf wurde festgestellt, dass der Navigationservice von Prism Forms um eine Methode namens \texttt{SelectTab()} erweitert wurde \cite{prism-selecttab}. Jedoch ist diese Extensionmethode noch nicht im stabilen Release von Prism Forms und daher noch nicht nutzbar. In einer Folgearbeit ist es schwer zu empfehlen, diese Funktionalität zu testen und zu verwenden. 

