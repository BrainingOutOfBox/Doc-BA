\subsection{Anforderungsspezifikation}
In dieser Sektion werden die Ergebnisse der Anforderungsanalyse in Form von diversen Diagrammen und Tabellen festgehalten. Ziel ist es, nötige Einschränkungen für das System zu definieren während genug Spielraum für die Umsetzung gelassen wird. 

\subsubsection{Funktionale Anforderungen}
Aufgrund der Fortsetzung der Arbeit existieren viele Überlappungen. Im Folgenden werden die Gemeinsamkeiten und Unterschiede beschrieben und detailliert auf Neuigkeiten eingegangen. 


\begin{figure}[h]
	\centering
	\includegraphics[width=1\linewidth]{./img/anforderungen/UC-Methode635.png}
	\caption{Use-Case Diagramm Methode 635}
	\label{fig:uc-methode635}
\end{figure}
Abbildung \ref{fig:uc-methode635} zeigt das Use-Case Diagramm für diese Arbeit. Es gibt einige Abweichungen zur Studienarbeit. Im Diagramm sind einige Use-Cases in blasser Farbe, bei diesen handelt es sich um bereits implementierte Features im Umfang der Studienarbeit. 

Wichtig zu erwähnen ist dabei, dass UC12: Create Brainstorming Finding erst im Nachhinein hinzugekommen ist. Dies ist so, weil zu Beginn der Entwicklung wir die Annahme trafen, dass für jedes Team genau ein Brainstorming entsteht und zwar zum Zeitpunkt des Starts des Brainstorming Findings (UC9). Wie wir aber im Verlaufe der Zeit feststellten, war die Umsetzung mit mehreren Brainstorming Findings pro Team keinen grossen Aufwand. So entstand ein zusätzlicher Use-Case, der nicht mehr implizit mit UC9: Start Brainstorming Finding ausgeführt wurde.

Weiter sind Use-Cases in satter Farbe erkennbar. Bei diesen handelt es sich um diejenigen, welche in dieser Arbeit im Fokus liegen. Komplett neu hinzugekommen sind UC8e: Insert Pattern und UC8f: Insert Video. Als zentralen Use-Case wird zudem UC8c: Insert Sketch angesehen. Im folgenden Kapitel werden nun die in dieser Arbeit relevanten Use-Cases kurz beschrieben. Im Kapitel \ref{par:fully-dressed-uc} sind alle zu implementierenden Sub-Use-Cases von UC8: Create Brainwave im fully-dressed Stil detailliert erläutert.
\paragraph{Brief Use-Cases}

\begin{basedescript}{
		\desclabelstyle{\multilinelabel}
		\desclabelwidth{4.5cm}
		\setlength{\itemsep}{5ex}}
	
	\item[\textit{UC2: }Logout] Als eingeloggter Participant will ich mich ausloggen, sodass das Startfenster wieder erscheint.
	
	\item[\textit{UC6: }Leave Brainstorming Team] Als Participant will ich ein beigetretenes Team verlassen können.
		
	\item[\textit{UC8: }Create Brainwave] Als Participant will ich während einer Brainstorming Session ein Brainwave (bestehend aus mehreren Ideen) erstellen und einreichen können. 
	
	\item[\textit{UC8a: }Insert Weblink] Als Participant will ich einen Weblink in mein aktuelles Sheet einfügen können.
	
	\item[\textit{UC8b: }Insert Picture] Als Participant will ich ein Bild in mein aktuelles Sheet einfügen können.
	
	\item[\textit{UC8c: }Insert Sketch] Als Participant will ich in mein aktuelles Sheet zeichnen können.
	
	\item[\textit{UC8e: }Insert Pattern] Als Participant will ich aus Vorlagen software-relevante Patterns in mein aktuelles Sheet einfügen können.
	
	\item[\textit{UC8f: }Insert Video] Als Participant will ich ein Video aufnehmen und in mein aktuelles Sheet einfügen können.
	
	\item[\textit{UC11: }Delete Brainstorming Team] Als Moderator will ich ein Brainstorming Team löschen können.
\end{basedescript}
\vspace{1cm}

\paragraph{Fully-Dressed Use-Cases}\label{par:fully-dressed-uc}


\paragraph{Abuse-Cases}
\paragraph{Sequenzdiagramm}

\subsubsection{Nicht-Funktionale Anforderungen}
Wie schon in unserer Studienarbeit halten wir uns auch hier wieder an die Standards ISO 9126\cite{ISO9126} bzw. dessen Nachfolger ISO 25010\cite{ISO9126_ISO25010}. Beide ISO-Normen sind sich sehr ähnlich und liefern eine gute Checkliste für jegliche Art von Systemanforderungen.

\begin{figure}[h]
	\centering
	\includegraphics[width=1\linewidth]{img/anforderungen/quality}
	\caption[Anforderungskategorien nach ISO 25010]{Anforderungskategorien nach  ISO 25010}\cite{ISO25010_Bild}
	\label{fig:ISO 25010}
\end{figure}

Im Gegensatz zur Studienarbeit wollen wir uns bei dieser Arbeit aber vor allem auf Faktoren wie Erweiterbarkeit und Modularität konzentrieren. Auch sollen Faktoren wie Zeitverhalten und Ästhetik stärker in den Vordergrund rücken. 

Die bekannten nicht-funktionale Anforderungen aus der Studienarbeit bleiben allerdings weiter bestehen. Um genaue und erfüllbare nicht-funktionale Anforderungen zu definieren, müssen die SMART-Kriterien \cite{SMART} erfüllt sein.

\begin{center}
    \begin{tabular}{ | p{6cm} | p{2.5cm} | p{2.5cm} | p{2.5cm} |}
    	\hline
    Kriterium & Minimum & Optimal & Übertroffen \\ 
    	\hline
    \textbf{Zeitverhalten} \newline Zeitkritische Kommunikation (Abgabe von Ideen/Brainsheets) zwischen Server und App beträgt: & 2 Sekunden & 1 Sekunden & weniger als 1 Sekunde \\
    	\hline
    \textbf{Erweiterbarkeit} \newline Die Anzahl an neuen Modularten (Problem-Arten), welche mit der bestehenden Architektur als umsetzbar gelten, wird als ... angesehen: & zu wenig & ausreichend & unbegrenzt \\
    	\hline
    \textbf{Modularität} \newline Die App ist innert ... Tagen um ein neues Modul (Problem-Art) erweitert: & 5 Tage & 2 Tage & weniger als 1 Tag \\
    	\hline
    \textbf{Ästhetik} \newline Die Anziehungskraft gegenüber dem Endnutzer wird als ... charakterisiert: & gering & in Ordnung & süchtig \\
    	\hline
    \textbf{Ausgereiftheit} \newline Der Grad der Ausgereiftheit oder Reife wird als ... angesehen: & ungenügend für den produktiven Einsatz (Prototypen-Stadium) & genügend für den produktiven Einsatz, kann aber immer noch in Fehlerzustände gelangen & kaum Fehlerzustände und somit keine Abstürze der App  \\
    	\hline
    \end{tabular}
\end{center}
