\subsection{Schlussfolgerungen}
Im Kapitel der Schlussfolgerungen wollen wir nochmals auf unser Projekt und dessen Verlauf schauen und unsere Ergebnisse kritisch bewerten. Dabei wollen wir aufzeigen, was wir in dieser Zeit erreicht haben, aber auch an welchen Stellen es noch Verbesserungspotenzial gibt. Ausserdem soll im Kapitel \ref{subsub:Ausblick} aufgezeigt werden, was das weitere Vorgehen für dieses Projekt sein könnte.

\subsubsection{Ergebnisbewertung}

\subsubsection{Bekannte Probleme}

\subsubsection{Ausblick}
\label{subsub:Ausblick}
Da die Ausbaumöglichkeiten dieses Projektes sehr vielfältig sind, empfinden wir es als sehr lohnenswert diese Applikation zu erweitern.

Als mögliche Erweiterung ist es vorstellbar, die Methode 'World-Cafe' \cite{world-cafe} in die bestehende Applikation zu integrieren. Im Gegensatz zur Methode 635 sind hier nicht einzelne Teilnehmer, welche ihre Ideen zu einem bestimmten Problem schildern sondern ganze Gruppen.

Das Konzept von World-Cafe sieht daher vor, dass man sich in einer Gruppe zu einer bestimmten Fragestellung austauscht. Ist eine vordefinierte Zeitspanne abgelaufen, wechseln die Teilnehmer zu einer anderen Gruppe, um dort wieder eine andere oder auch die gleiche Fragestellung zu diskutieren. 

Um die World-Cafe Methode in die bestehende Applikation zu integrieren, ist es an dieser Stelle aber ratsam von der ursprünglichen Version abzuweichen und die Gruppen immer gleich zu belassen. 

Das würde dann bedeuten, dass mehrere Teams (statt mehrere Participants) gemeinsam ein BrainstormingFinding erarbeiten. Der Umstand dass mehrere Teams ein BrainstormingFinding lösen führt dazu, dass innerhalb eines Teams schneller verschiedenste Ideen entstehen, als wenn ein Participant alleine Ideen/Lösungsansätze entwickelt. 

Dies würde der gesamten Applikation nochmals einen Mehrwert bieten. 