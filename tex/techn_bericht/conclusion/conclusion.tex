\subsection{Schlussfolgerungen}
\label{subsec:conclusions}
Im Kapitel der Schlussfolgerungen wollen wir nochmals auf unser Projekt und dessen Verlauf schauen und unsere Ergebnisse kritisch bewerten. Dabei wollen wir aufzeigen, was wir in dieser Zeit erreicht haben, aber auch an welchen Stellen es noch Verbesserungspotenzial gibt. Ausserdem soll das Kapitel \ref{subsub:Ausblick} veranschaulichen, was das weitere Vorgehen für dieses Projekt sein könnte.

\subsubsection{Ergebnisbewertung}
\label{subsec:ergebnisbewertung}
%TODO Technologien reflektieren?
%Wahl der Technologien war gut. Möglichkeiten (Schemalos) von MongoDB haben uns sicherlich Zeit gespart.
% Asynchroner MongoDB Treiber vlt. nicht notwenig. Mit dem synchronen Treiber hätte es sicherlich auch funktioniert.
%no security
%dank frühen User-Tests konnte noch Feedback eingebaut werden.

%%Evtl. kann das Ergebnis rausgenommen werden
%Als Ergebnis der vorliegenden Bachelorarbeit können wir eine lauffähige, stabile und performante Cross-Plattform Applikation für Android und iOS präsentieren. Diese ermöglicht es Benutzern die Methode 635 auf ihrem Smartphone oder Tablet anzuwenden. Neben den bereits aus der Studienarbeit verfügbaren Funktionen, wie dem Erfassen von Textideen, ist es nun zusätzlich möglich, Skizzen als Teil des Brainstormings zu zeichnen oder aus einer Liste von Pattern ein passendes Pattern auszuwählen. Des weiteren wurde eine Export-Funktion implementiert, welche es dem Benutzer ermöglicht, ein durchgeführtes BrainstormingFinding auch ausserhalb unserer Applikation bei Bedarf weiter zu bearbeiten.

Als der Teil nicht-funktionalen Anforderungen im Kapitel \ref{subsub:nfr} haben wir sogenannte Landing Zones für die in diesem Projekt wichtigen SMART-Kriterien definiert. Wir wollen hier nochmals speziell Bezug nehmen auf diese Ziele und kritisch beurteilen, wie gut diese in unserem Projekt umgesetzt wurden.
 
 Mehrere User-Tests haben gezeigt, dass das Zeitverhalten bzw. die zeitkritische Kommunikation zwischen Server und App in einem akzeptablen Rahmen von geschätzt einer Sekunde liegt. Dies kommt aber auch auf die Netzwerkkonnektivität an. 
 
Wir beurteilen dieses Kriterium als 'gut' erfüllt (vgl. Kapitel \ref{subsub:nfr}).
 
 In Punkto Erweiterbarkeit sind wir der Meinung, dass sich  noch viele weitere Ideenarten für eine mögliche Integration eignen würden. Im Kapitel \ref{subsub:Ausblick} haben wir dafür drei konkrete Ideenarten aufgeschrieben. Zudem gibt es Ideenarten, wie z.B. die PictureIdea oder VideoIdea, welche wir zwar zu Beginn dieser Arbeit definiert hatten aber mangels Zeit nicht umsetzen konnten. 
 
Das Ziel der Erweiterbarkeit sehen wir daher als 'übertroffen' an (vgl. Kapitel \ref{subsub:nfr}).

Dank der Anleitung in Anhang \ref{sec:Ideen_Erweiterung} sind wir zudem der Ansicht, dass sich die Applikation innert wenigen Tagen um eine neue Ideenart erweitern liesse. In Zahlen ausgedrückt, sollte sich solch ein Vorhaben schätzungsweise innert zwei Tagen realisieren lassen. 

Das Ziel der Modularität ist somit 'gut' erreicht (vgl. Kapitel \ref{subsub:nfr}).

Die Frage der Ästhetik ist als Kriterium eher schwierig zu beurteilen, zumal jede Person wahrscheinlich etwas andere Vorstellungen diesbezüglich hat. Dennoch gehen wir davon aus, dass es für die Mehrheit der Endnutzer eine nachhaltige Anziehungskraft hätte. An einem der zwei durchgeführten User-Test wurde sogar explizit erwähnt, dass die App ansprechend aussieht. Die User-Tests haben aber auch aufgezeigt, dass die App in Sachen Orientierung und die Usability noch Potenzial für Verbesserungen aufweist.

Die Ästhetik der App beurteilen wir insgesamt als 'gut' (vgl. Kapitel \ref{subsub:nfr}).

Den Grad der Ausgereiftheit ist unserer Meinung nach als 'genügend für den produktiven Einsatz' zu beurteilen. Auch hier haben User-Tests gezeigt, dass es, trotz der massiven Verbesserungen zur Studienarbeit, immer noch vorkommen kann, dass die Applikation aus unbekannten Gründen abstürzt. In solch einem Fall ist es aber ohne Probleme möglich, sich wieder einzuloggen und weiterzufahren. Dem produktiven Einsatz steht somit nichts im Wege.

Das Ziel der Ausgereiftheit ist somit auch als 'gut' zu beurteilen (vgl. Kapitel \ref{subsub:nfr}).

\begin{center}
    \begin{tabular}{ | p{6cm} | p{2.5cm} | p{2.5cm} | p{2.5cm} |}
    	\hline
    Kriterium & Minimum & Gut & Übertroffen \\ 
    	\hline
    \textbf{Zeitverhalten} \newline Zeitkritische Kommunikation (Abgabe von Ideen/Brainsheets) zwischen Server und App beträgt & 2 Sekunden & \cellcolor{green!25} 1 Sekunden & weniger als 1 Sekunde \\
    	\hline
    \textbf{Erweiterbarkeit} \newline Die Anzahl an neuen Ideenarten, welche mit der bestehenden Architektur als umsetzbar gelten, wird als ... angesehen & zu wenig (0-1) & ausreichend (1-5) & \cellcolor{green!25} unbegrenzt ($>$5)\\
    	\hline
    \textbf{Modularität} \newline Die App ist innert ... Tagen um eine neue Ideenart erweitert & 5 Tage & \cellcolor{green!25} 2 Tage & weniger als 1 Tag \\
    	\hline
    \textbf{Ästhetik} \newline Die Anziehungskraft gegenüber dem Endnutzer wird als ... charakterisiert & ausreichend für eine kritische Masse an Nutzern & \cellcolor{green!25} nachhaltig & tägliche Nutzung in kritischen Projekten \\
    	\hline
    \textbf{Ausgereiftheit} \newline Der Grad der Ausgereiftheit oder Reife wird als ... angesehen & ungenügend für den produktiven Einsatz (Prototypen-Stadium) & \cellcolor{green!25} genügend für den produktiven Einsatz, kann aber immer noch in Fehlerzustände gelangen & kaum Fehlerzustände und somit keine Abstürze der App  \\
    	\hline
    \end{tabular}
\end{center}

\subsubsection{Bekannte Probleme}
Dieses Kapitel dient dazu Fehlverhalten in unserem System zu dokumentieren.

\begin{basedescript}{
		\desclabelstyle{\multilinelabel}
		\desclabelwidth{3.5cm}
		\setlength{\itemsep}{5ex}}
	\item [Falscher Zustand nach erstmaligem Auswählen eines beendeten Brainstormings] Wird nach einem beendeten Brainstorming zurück in die Übersicht aller Brainstormings gewechselt und das durchgeführte ausgewählt, wird der Zustand der Statemachine falsch eruiert (\texttt{RunningState} anstatt \texttt{EndedState}). Dies führt dazu, dass der Timer weiterläuft und schlussendlich ins Negative zählt. Dies hängt damit zusammen, dass der \texttt{Brainstorming\-Service} durch das mehrmalige Aufrufen der ViewModels aufgrund des Navigationsproblems (siehe Kapitel \ref{subsub:navigation}) im  Kontextobjekt veraltete Werte enthält, welche in der Berechnung der Statemachine in einem falschen Zustand resultieren. 
	
	\item [Security] Obwohl das Thema Security kein eigentliches Problem darstellt, soll an dieser Stelle erwähnt werden, dass wir kein spezielles Augenmerk auf dieses Thema geworfen haben. Bei einer allfälligen Folgearbeit wäre es daher sinnvoll, dieses Thema stärker in den Fokus zu rücken und notwendige Anpassungen an der bestehenden Applikation vorzunehmen. Auch wäre es ratsam sich Gedanken für den Zugriff von Skizzen zu machen, welche möglicherweise als intern oder schützenswert zu klassifizieren sind.
\end{basedescript}


\subsubsection{Ausblick}
\label{subsub:Ausblick}
Da die Ausbaumöglichkeiten dieses Projektes sehr vielfältig sind, empfinden wir es als sehr lohnenswert diese Applikation zu erweitern.

Dazu sehen wir drei mögliche Arten oder Wege, wie die bestehende Applikation verbessert bzw. erweitert werden könnte. Diese wären weitere Ideen-Typen zu integrieren, die erarbeiteten Ideen stärker miteinander zu referenzieren oder die Methode 635 mit der World-Cafe Methode zu kombinieren.

\paragraph*{Weitere Ideen-Typen}~\\
Die einfachste und naheliegenste Erweiterung könnte darin bestehen, die Applikation um weitere Ideen-Typen zu erweitern. Die gesamte Applikation ist so programmiert, dass dies mit möglichst wenig Aufwand machbar ist (siehe Anhang \ref{sec:Ideen_Erweiterung}). 

Dabei könnten wir uns vorstellen, dass sich Ideen-Typen wie QualityIdea (angelehnt an nicht-funktionale Anforderungen), RequirementIdea (angelehnt an funktionale Anforderungen)  oder CodeIdea gut umsetzen liessen.

\paragraph*{Ideen stärker referenzieren}~\\
Bei verschiedensten Durchführungen der Methode 635 auf dem Papier, kam immer wieder die Frage auf, wie ich den anderen Teilnehmern mitteilen kann, auf welche Idee ich gerade Bezug nehme. Dies ist in der vorliegenden Applikation nicht möglich bzw. nur indem man sich darauf einigt, dass die Reihenfolge der Ideen ausschlaggebend ist.

Eine weitere Verbesserung würde demnach darin bestehen, dem Endnutzer die Möglichkeit zu bieten seine Ideen stärker mit den vorliegenden Ideen zu 'verknüpfen' oder zu referenzieren. Dies würde auch das Problem der oben geschilderten Unsicherheit lösen.

\paragraph*{Methode World-Cafe}~\\
Als weitere mögliche Erweiterung ist es vorstellbar, die Methode 'World-Cafe' \cite{world-cafe} in die bestehende Applikation zu integrieren. Im Gegensatz zur Methode 635 sind hier nicht einzelne Teilnehmer, welche ihre Ideen zu einem bestimmten Problem schildern, sondern ganze Gruppen.

Das Konzept von World-Cafe sieht daher vor, dass man sich in einer Gruppe zu einer bestimmten Fragestellung austauscht. Ist eine vordefinierte Zeitspanne abgelaufen, wechseln die Teilnehmer zu einer anderen Gruppe, um dort wieder eine andere oder auch die gleiche Fragestellung zu diskutieren. 

Um die World-Cafe Methode in die bestehende Applikation zu integrieren, ist es an dieser Stelle aber ratsam von der ursprünglichen Version abzuweichen und die Gruppen immer gleich zu belassen. 

Das hätte zur Folge, dass mehrere Teams (statt mehrere Participants) gemeinsam ein BrainstormingFinding erarbeiten. Der Umstand, dass mehrere Teams ein BrainstormingFinding lösen, führt dazu, dass innerhalb eines Teams schneller verschiedenste Ideen entstehen, als wenn ein Participant alleine Ideen/Lösungsansätze entwickelt. 

Dies würde der gesamten Applikation nochmals einen Mehrwert bieten. 
