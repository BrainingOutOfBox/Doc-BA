\subsection{Schlussfolgerungen}
Im Kapitel der Schlussfolgerungen wollen wir nochmals auf unser Projekt und dessen Verlauf schauen und unsere Ergebnisse kritisch bewerten. Dabei wollen wir aufzeigen, was wir in dieser Zeit erreicht haben, aber auch an welchen Stellen es noch Verbesserungspotenzial gibt. Ausserdem soll im Kapitel \ref{subsub:Ausblick} aufgezeigt werden, was das weitere Vorgehen für dieses Projekt sein könnte.

\subsubsection{Ergebnisbewertung}

\subsubsection{Bekannte Probleme}

\subsubsection{Ausblick}
\label{subsub:Ausblick}
Da die Ausbaumöglichkeiten dieses Projektes sehr vielfältig sind, empfinden wir es als sehr lohnenswert diese Applikation zu erweitern.

Dazu sehen wir drei mögliche Arten oder Wege, wie die bestehende Applikation verbessert bzw. erweitert werden könnte. Diese wären weitere Ideen-Typen zu integrieren, die erarbeiteten Ideen stärker miteinander zu referenzieren oder die Methode 635 mit der World-Cafe Methode zu kombinieren.

\paragraph*{Weitere Ideen-Typen}~\\
Die einfachste und naheliegenste Erweiterung könnte darin bestehen, die Applikation um weitere Ideen-Typen zu erweitern. Die gesamte Applikation ist so programmiert, dass dies mit möglichst wenig Aufwand machbar ist (siehe Anhang \ref{sec:Ideen_Erweiterung}). 

Dabei könnten wir uns vorstellen, dass sich Ideen-Typen wie QualityIdea (angelehnt an nicht-funktionale Anforderungen), RequirementIdea (angelehnt an funktionale Anforderungen)  oder CodeIdea gut umsetzen liessen.

\paragraph*{Ideen stärker referenzieren}~\\
Bei verschiedensten Durchführungen der Methode 635 auf dem Papier, kam immer wieder die Frage auf, wie ich den anderen Teilnehmern mitteilen kann, auf welche Idee ich gerade Bezug nehme. Dies ist in der vorliegenden Applikation nicht möglich bzw. nur indem man sich darauf einigt, dass die Reihenfolge der Ideen ausschlaggebend ist.

Eine weitere Verbesserung würde demnach darin besteh, dem Endnutzer die Möglichkeit zu bieten seine Ideen stärker mit den vorliegenden Ideen zu 'verknüpfen' oder zu referenzieren. Dies würde auch das Problem der oben geschilderten Unsicherheit lösen.

\paragraph*{Methode World-Cafe}~\\
Als weitere mögliche Erweiterung ist es vorstellbar, die Methode 'World-Cafe' \cite{world-cafe} in die bestehende Applikation zu integrieren. Im Gegensatz zur Methode 635 sind hier nicht einzelne Teilnehmer, welche ihre Ideen zu einem bestimmten Problem schildern, sondern ganze Gruppen.

Das Konzept von World-Cafe sieht daher vor, dass man sich in einer Gruppe zu einer bestimmten Fragestellung austauscht. Ist eine vordefinierte Zeitspanne abgelaufen, wechseln die Teilnehmer zu einer anderen Gruppe, um dort wieder eine andere oder auch die gleiche Fragestellung zu diskutieren. 

Um die World-Cafe Methode in die bestehende Applikation zu integrieren, ist es an dieser Stelle aber ratsam von der ursprünglichen Version abzuweichen und die Gruppen immer gleich zu belassen. 

Das hätte zur Folge, dass mehrere Teams (statt mehrere Participants) gemeinsam ein BrainstormingFinding erarbeiten. Der Umstand dass mehrere Teams ein BrainstormingFinding lösen führt dazu, dass innerhalb eines Teams schneller verschiedenste Ideen entstehen, als wenn ein Participant alleine Ideen/Lösungsansätze entwickelt. 

Dies würde der gesamten Applikation nochmals einen Mehrwert bieten. 
