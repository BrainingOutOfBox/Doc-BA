\subsection{Ergebnisse}
Das Kapitel der Ergebnisse befasst sich mit der konkreten Umsetzung der gesamten Applikation und dokumentiert die grundlegenden Ergebnisse oder Resultate, welche aus dieser Arbeit hervorgegangen sind. 

Auch hier soll nochmals erwähnt sein, dass diese Arbeit auf der Studienarbeit \cite{methode635-sa} aufbaut und grundlegende Informationen dort nachzulesen sind. 

In diesem Kapitel gehen wir zunächst auf die Änderungen während der Refactoring-Phase ein und schildern was dies für die Wartung und Erweiterung des Codes bedeutet. 

Anschliessend sind weitere Features dokumentiert, welche im Verlauf dieser Arbeit an der Applikation durchgeführt wurden.

Das Kapitel wird durch eine Gegenüberstellung der erreichten und geplanten Arbeit abgeschlossen.

Zur besseren Übersicht wurde der Code vereinzelt gekürzt. Dies ist durch 3 Punkte (...) gekennzeichnet.

\subsubsection{Refactoring zu Beginn der Arbeit}
Wie aus dem Projektplan (siehe Abbildung \ref{fig:projekt-plan}) zu entnehmen ist, haben wir uns zu Beginn unserer Arbeit dazu entschieden, zwei Sprints für das Refactoring unsere Applikation zu nutzen. Die Entscheidung überhaupt ein Refactoring durchzuführen entstand aus der Tatsache, dass wir schon gegen Ende der Studienarbeit gemerkt hatten, dass bei einer allfälligen Weiterführung der Arbeit ein Refactoring von grossem Wert sein wird. Auch haben wir diesbezüglich schon während der Studienarbeit einzelne Vorschläge für ein Refactoring beschrieben.


Das Ziel der Refactoring-Phase war es daher, den Code welcher aus der Studienarbeit hervorgegangen ist, zu prüfen und zu verbessern. Dies sollte uns helfen, die Erweiterbarkeit für zukünftig geplante Features zu gewähren. Für die Überprüfung konnte Silvan Gehrig gewonnen werden. Seine Rückmeldungen, welche im Anhang \ref{sec:code-review} nachgelesen werden können, wurden wie nachfolgend beschrieben umgesetzt.

Auch konnten wir die in der Studienarbeit beschriebenen Vorschläge erfolgreich in der App umsetzen.
 
\paragraph*{Refactoring Backend}~\\
Dieser Abschnitt ist so aufgebaut, dass anhand von zwei Beispielen aufgezeigt werden soll, wie der Code im Backend vor und nach dem Refactoring ausgesehen hat. Auch wird am Ende des Abschnitts kurz erwähnt, was dies für Verbesserungen zur Folge hatte.

Das nachfolgende Listing \ref{participantErstellenVorRef} zeigt einen Ausschnitt aus der \texttt{Participant\-Controller\-.java} Klasse wie es vor dem Refactoring war. Das Listing \ref{participantErstellenNachRef} zeigt dieselbe Methode nach dem Refactoring.

\lstset{language=JAVA, showstringspaces=false, frame=single, captionpos=b, label=createParticipant, breaklines=true, numbers=left}
\begin{lstlisting}[caption={Participant erstellen vor Refactoring}, label=participantErstellenVorRef]
public Result createParticipant(){

    JsonNode body = request().body().asJson();

    if (body == null ) {
        return forbidden(Json.toJson(new ErrorMessage("Error", "json body is null")));
    } else if(  body.hasNonNull("username") &&
            	body.hasNonNull("password") &&
            	body.hasNonNull("firstname") &&
            	body.hasNonNull("lastname")) {

    Participant participant = new Participant(body.get("username").asText(), body.get("password").asText(), body.get("firstname").asText(), body.get("lastname").asText());

    participantCollection.insertOne(participant, new SingleResultCallback<Void>() {
        @Override
        public void onResult(Void result, Throwable t) {
            Logger.info("Inserted Participant!");
        }
    });

    return ok(Json.toJson(new SuccessMessage("Success", "Participant successfully inserted")));
    }

    return forbidden(Json.toJson(new ErrorMessage("Error", "json body not as expected")));
}
\end{lstlisting}

\begin{lstlisting}[caption={Participant erstellen nach Refactoring}, label=participantErstellenNachRef]
@BodyParser.Of(ParticipantDTOBodyParser.class)
public Result createParticipant() {
    ParticipantDTO participantDTO = request().body().as(ParticipantDTO.class);
    Participant participant = modelsMapper.toParticipant(participantDTO);

    try {
        if (service.insertParticipant(participant)){
            return ok(Json.toJson(new SuccessMessage("Success", "Participant successfully inserted")));
        } else {
            Logger.info("Username already exists");
            return badRequest(Json.toJson(new ErrorMessage("Error", "Username already exists")));
        }
    } catch (ExecutionException | InterruptedException e) {
        return internalServerError(Json.toJson(new ErrorMessage("Error", e.getMessage())));
    }
}
\end{lstlisting}

Das Erstellen des Participants wurde komplett in den \texttt{Participant\-DTO\-Body\-Parser\-.class} ausgelagert. Dieser erstellt nun aus dem angelieferten JSON (JavaScript Object Notation) ein Data-Transfer-Object \cite{DTO}, welches in einem nächsten Schritt zu einem Business Object aufgelöst wird (Zeile 5).

Jegliche Überprüfungen aus Listing \ref{participantErstellenVorRef} (Zeilen 5 und 7-10) konnten so zentral in den BodyParser ausgelagert werden.

Bei den nachfolgenden Listings \ref{putBrainsheetVorRef} und \ref{putBrainsheetNachRef}, welche einen Ausschnitt aus dem \texttt{Brain\-storming\-Finding\-Controller\-.java} zeigen, ist der Unterschied noch stärker zu sehen. So konnte zum Beispiel die \texttt{putBraisheet} Methode von zirka 40 Zeilen auf 15 Zeilen gekürzt werden. 

\begin{lstlisting}[caption={PutBrainsheet vor Refactoring}, label=putBrainsheetVorRef]
public Result putBrainsheet(String findingIdentifier) throws ExecutionException, InterruptedException {

JsonNode body = request().body().asJson();
JsonNode brainwaves = body.findPath("brainwaves");
JsonNode nrOfSheet = body.findPath("nrOfSheet");

if (body == null ) {
    return forbidden(Json.toJson(new ErrorMessage("Error", "json body is null")));
} else if(  !brainwaves.isNull() &&
            !nrOfSheet.isNull()){

	BrainstormingFinding finding = getBrainstormingFinding(findingIdentifier);

    if (finding == null){
        return internalServerError(Json.toJson(new ErrorMessage("Error", "No brainstormingFinding found")));
    }

    Brainsheet oldBrainsheet = finding.getBrainsheets().get(nrOfSheet.asInt());
    Brainsheet newBrainsheet = createBrainsheet(body);


    findingCollection.updateOne(eq("identifier", findingIdentifier),pullByFilter(Filters.eq("brainsheets", oldBrainsheet)), new SingleResultCallback<UpdateResult>() {
        @Override
        public void onResult(final UpdateResult result, final Throwable t) {
            Logger.info(result.getModifiedCount() + " Brainsheet successfully deleted");
        }
    });

    findingCollection.updateOne(eq("identifier", findingIdentifier),combine(pushEach("brainsheets", Arrays.asList(newBrainsheet), new PushOptions().position(newBrainsheet.getNrOfSheet())), inc("deliveredBrainsheetsInCurrentRound", 1)), new SingleResultCallback<UpdateResult>() {
        @Override
        public void onResult(final UpdateResult result, final Throwable t) {
            Logger.info(result.getModifiedCount() + " Brainsheet successfully inserted");
        }
    });

    return ok(Json.toJson(new SuccessMessage("Success", "Brainsheet successfully updated")));
}

return forbidden(Json.toJson(new ErrorMessage("Error", "json body not as expected")));
}
\end{lstlisting}


\begin{lstlisting}[caption={PutBrainsheet nach Refactoring}, label=putBrainsheetNachRef]
@BodyParser.Of(BrainsheetDTOBodyParser.class)
public Result putBrainsheet(String findingIdentifier){
    BrainsheetDTO brainsheetDTO = request().body().as(BrainsheetDTO.class);
    Brainsheet newBrainsheet = modelsMapper.toBrainsheet(brainsheetDTO);

    try {

        if (service.exchangeBrainsheet(findingIdentifier, newBrainsheet)) {
            return ok(Json.toJson(new SuccessMessage("Success", "Brainsheet successfully updated")));
        } else {
            return badRequest(Json.toJson(new ErrorMessage("Error", "No Brainsheet updated")));
        }

    } catch (ExecutionException | InterruptedException e) {
        return internalServerError(Json.toJson(new ErrorMessage("Error", e.getMessage())));
    }
}
\end{lstlisting}

Die gesamte Logik für den Austausch der Brainsheets wurde in die \texttt{FindingService\-.java} Klasse ausgelagert.

\begin{lstlisting}[caption={Exchange Brainsheet im Business Layer}, label=exchangeBrainsheetBusinessLayer]
public boolean exchangeBrainsheet(String findingIdentifier, Brainsheet newBrainsheet) {
BrainstormingFinding finding = service.getFinding(findingIdentifier).get();

if (finding == null){
    return false;
} else {
    if (newBrainsheet.getNrOfSheet() < finding.getBrainsheets().size()) {
        Brainsheet oldBrainsheet = finding.getBrainsheets().get(newBrainsheet.getNrOfSheet());
        service.exchangeBrainsheet(finding, oldBrainsheet, newBrainsheet);
        return true;
    }
    return false;
}

}
\end{lstlisting}

Auch wurde die Implementation für das Einfügen in die Datenbank in eine separate Klasse (\texttt{MongoDBFindingservice.java}) ausgelagert, um eine bessere Übersicht und ein besseres Layering zu gewähren.

\begin{lstlisting}[caption={Exchange Brainsheet im Data Access Layer}, label=exchangeBrainsheetDAL]
public void exchangeBrainsheet(BrainstormingFinding finding, Brainsheet oldBrainsheet, Brainsheet newBrainsheet){
    
    //delete old Brainsheet
    findingCollection.updateOne(eq("identifier", finding.getIdentifier()),pullByFilter(Filters.eq("brainsheets", oldBrainsheet)), (result, t) -> Logger.info(result.getModifiedCount() + " old Brainsheet successfully deleted"));
    
    //insert new Brainsheet at the same place
    findingCollection.updateOne(eq("identifier", finding.getIdentifier()),combine(pushEach("brainsheets", Arrays.asList(newBrainsheet), new PushOptions().position(newBrainsheet.getNrOfSheet())), inc("deliveredBrainsheetsInCurrentRound", 1)), (result, t) -> Logger.info(result.getModifiedCount() + " new Brainsheet successfully inserted"));
}
\end{lstlisting}
Diese zwei Beispiele stellen noch lange nicht alle überarbeiteten Codestellen dar, geben aber eine gute Übersicht, wie der gesamte Code vereinfacht werden konnte.

\paragraph*{Fazit}
Gerade das Beispiel vom Austausch der Brainsheets (Listing \ref{putBrainsheetVorRef}) verdeutlicht, dass das Layering stark verbessert wurde. Vieles was vorher in einer Klasse war, konnte in verschiedene Klassen ausgelagert werden. Damit konnte die Übersicht und Komplexität deutlich verbessert werden.

Auch wurden im gesamten Backend Data-Transfer-Obekte (DTO) \cite{DTO} eingefügt. Dies ermöglicht eine stärkere Trennung von relevanten Informationen für Business-Objekte (BO) und relevanten Informationen für Data-Transfer-Obekte. Dies war vor allem im späteren Verlauf der Arbeit von grossem Wert.

Mit Hilfe der BodyParser-Klassen konnte die komplette Überprüfung und Deserialisierung der angelieferten JSON-Daten zentral geregelt werden. Somit ist immer sichergestellt, dass das DTO korrekt (alle erwarteten Informationen sind vorhanden und das Format stimmt) erstellt wird. 

\paragraph*{Refactoring Frontend}~\\

\subsubsection{Implementation Skizzen Feature}
Nachdem die Refactoring-Phase beendet war, begannen wir die Skizzen-Funktion in unsere Applikation zu integrieren. Diese Funktion sollte es dem Endnutzer ermöglichen, nebst den \texttt{NoteIdeas} auch \texttt{SketchIdeas} (siehe Abbildung \ref{fig:domainmodell-methode635}) zu erfassen. Mit den \texttt{SketchIdeas} wird dem Endnutzer die Möglichkeit gegeben, Skizzen als Teil der Brainstorming-Runden zu zeichnen.

Da es sich bei den einzelnen Skizzen um Bilder handelt, musste zunächst eine Möglichkeit geschaffen werden, diese Bilder in der Datenbank abzulegen. Dafür wurden in der Elaboration-Phase Analysen (siehe Kapitel \ref{seq:save_file_in_db}) durchgeführt und ein Prototyp erstellt, um die technische Machbarkeit zu verifizieren.

\paragraph*{Implementation Backend}~\\
In den nachfolgenden Listings wird aufgezeigt, wie die Sketch-Funktion im Backend umgesetzt wurde.
Der genaue Ablauf kann der Abbildung \ref{fig:Seq-Draw-Sketch} entnommen werden.

\begin{lstlisting}[caption={Upload File im File Controller}, label=uploadFileController]
@BodyParser.Of(MultipartFormDataBodyParser.class)
public Result uploadFile(){
    try {

    final Http.MultipartFormData<File> formData = request().body().asMultipartFormData();
    final Http.MultipartFormData.FilePart<File> filePart = formData.getFile("name");
    final File file = filePart.getFile();

    final byte[] fileData = Files.readAllBytes(file.toPath());
    final String fileName = file.getName();

    String fileId = service.uploadFileAsStream(fileData, fileName);
    return ok(Json.toJson(new SuccessMessage("Success", fileId)));

    } catch (IOException | InterruptedException | ExecutionException  e) {
        return internalServerError(Json.toJson(new ErrorMessage("Error", e.getMessage())));
    }
}
\end{lstlisting}

Ähnlich wie bei den JSON-Daten, wird auch hier zunächst das Bild mittels eines BodyParsers (Zeile 1) in ein File umgewandelt (Zeilen 5-7), welches dann als Byte-Array dem File-Service (Zeile 12) übergeben werden kann. 

\begin{lstlisting}[caption={Upload File im File Service}, label=uploadFileService]
public String uploadFileAsStream(byte[] stream, String fileName) ... {
    return service.uploadFileAsStream(stream,fileName);
}
\end{lstlisting}

Der File-Service nimmt diesen Stream entgegen und gibt diesen unverändert dem Datenbank-Service weiter.

\begin{lstlisting}[caption={Upload File im DB Service}, label=uploadFileDBService]
@Override
public String uploadFileAsStream(byte[] stream, String fileName) ... {
    ByteBuffer data = ByteBuffer.wrap(stream);
    CompletableFuture<String> future = new CompletableFuture<>();

    final GridFSUploadStream uploadStream = gridFSBucket.openUploadStream(fileName);
    uploadStream.write(data, (result, t) -> {
        Logger.info("File successfully inserted; ID: " + uploadStream.getObjectId().toHexString());
        future.complete(uploadStream.getObjectId().toHexString());

        uploadStream.close((result1, t1) -> {
            // stream close
        });
    });

    return future.get();
}
\end{lstlisting}

Der Datenbank-Service speichert anschliessend den Stream in die Datenbank (Zeile 7) und liefert bei erfolgreicher Speicherung die ObjektID zurück (Zeile 9). Am Ende wird noch der uploadStream zur Datenbank geschlossen (Zeile 11).

Die Smartphone-Applikation speichert nun die ObjektID als Teil der \texttt{SketchIdea} und sendet das \texttt{Brainsheet} wie gewohnt nach Ablauf der Zeit an das Backend.

Das Herunterladen der gespeicherten Bilder funktioniert auf ganz ähnliche Weise. Da die ObjektID in der \texttt{SketchIdea} abgelegt ist, kann das eigentliche Bild problemlos wiedergefunden werden.

\begin{lstlisting}[caption={Download File im DB Service}, label=uploadFileDBService]
@Override
public byte[] downloadFileAsStream(String id) ...{
    ObjectId fileId = new ObjectId(id);
    final ByteBuffer dstByteBuffer = ByteBuffer.allocate(1024 * 1024);
    final GridFSDownloadStream downloadStream = gridFSBucket.openDownloadStream(fileId);
    CompletableFuture<byte[]> future = new CompletableFuture<>();

    downloadStream.read(dstByteBuffer, (result, t) -> {
        dstByteBuffer.flip();
        byte[] bytes = new byte[result];
        dstByteBuffer.get(bytes);
        Logger.info("Found file to download; Size: " + result);
        future.complete(bytes);

        downloadStream.close((result1, t1) -> {
            // stream closed
        });
    });

    return future.get();
}
\end{lstlisting}

Hierbei wird ein downloadStream geöffnet, welcher die Daten anhand der ID aus der Datenbank liest und in ein Byte-Array schreibt (Zeile 8). Auch hier wird am Ende der Stream wieder geschlossen (Zeile 15).

Der zurückgelieferte Byte-Array wird anschliessend unverändert an den Client zurückgeschickt. Allfällige Fehler während der Ausführung sowohl beim Hochladen wie auch beim Herunterladen werden im File-Controller abgefangen und als Fehler an den Client geschickt.

\paragraph*{Implementation Frontend}~\\

\subsubsection{Implementation Pattern Feature}
Nebst der Skizzen-Funktion war es unser Ziel, dem Endnutzer eine Funktion anzubieten, welche es ihm ermöglicht, aus einer Liste von verschiedensten Pattern ein passendes Pattern auszuwählen. 

Wir erweiterten unsere Applikation daher um eine \texttt{PatternIdea}. Die verwendeten Pattern stammen ausschliesslich von der Webseite \href{https://microservice-api-patterns.org}{Microservice API Patterns}. Dabei ist es aber durchaus auch denkbar Pattern von anderen Quellen zu nehmen oder sich gar auf nicht-software Pattern zu konzentrieren.

\paragraph*{Implementation Backend}~\\
In den nachfolgenden Listings wird aufgezeigt, wie die Pattern-Funktion im Backend umgesetzt wurde.
Der genaue Ablauf kann der Abbildung \ref{fig:Seq-Insert-Pattern} entnommen werden.

\begin{lstlisting}[caption={Alle Pattern holen im Pattern Controller}, label=getAllPatternInController]
 public Result getAllPatternIdeas(){
    CompletableFuture<Queue<PatternIdea>> future = service.getAllPatternIdeas();
    ArrayList<JsonNode> list = new ArrayList<>();

    try {

        for (PatternIdea patternIdea : future.get()) {
            PatternIdeaDTO patternIdeaDTO = modelsMapper.toPatternIdeaDTO(patternIdea);
            JsonNode node = Json.toJson(patternIdeaDTO);
            ((ObjectNode)node).put("type", "patternIdea");
            list.add(node);
        }

        return ok(Json.toJson(list));

    } catch (InterruptedException | ExecutionException e) {
        return internalServerError(Json.toJson(new ErrorMessage("Error", e.getMessage())));
    }
}
\end{lstlisting}

Um alle verfügbaren Pattern abzurufen wird auf dem Pattern-Service (Zeile 2) die Methode \textit{getAllPatternIdeas} aufgerufen. Anschliessend werden die gefundenen Pattern lediglich in ein DTO umgewandelt (Zeile 8) und fast unverändert als JSON an den Client gesendet (Zeile 14).

Der Endnutzer kann nun aus einer Liste der verfügbaren Pattern das gewünschte Pattern auswählen. Dieses wird wie schon die \texttt{SketchIdea} oder \texttt{NoteIdea} als Teil des \texttt{Brainsheets} an das Backend geschickt. 

Wie in allen Controller-Klassen werden auch hier allfällige Fehler während der Ausführung abgefangen und an den Client gesendet (Zeile 16/17).

\begin{lstlisting}[caption={Alle Pattern holen im Pattern Service}, label=getAllPatternInService]
public CompletableFuture<Queue<PatternIdea>> getAllPatternIdeas(){
        return service.getAllPatternIdeas();
}
\end{lstlisting}

Der Pattern-Service leitet den Aufruf seinerseits direkt an den Datenbank-Service weiter.

\begin{lstlisting}[caption={Alle Pattern holen im Pattern DB Service}, label=getAllPatternInDBService]
@Override
public CompletableFuture<Queue<PatternIdea>> getAllPatternIdeas() {
    CompletableFuture<Queue<PatternIdea>> future = new CompletableFuture<>();
    Queue<PatternIdea> queue = new ConcurrentLinkedQueue<>();

    patternIdeaCollection.find().sort(Sorts.ascending("category")).forEach(
            finding -> queue.add(finding), (result, t) -> {
                Logger.info("Get all available patterns");
                future.complete(queue);
            });

    return future;
}
\end{lstlisting}

Der DB-Service wiederum findet alle Pattern und gibt diese nach Kategorie sortiert (Zeile 6-12) als Resultat zurück.

\paragraph*{Implementation Frontend}~\\

\subsubsection{Implementation Export Feature}
Die letzte Funktion, welche wir in unserer Bachelorarbeit verwirklichen wollten, war die Möglichkeit, die erarbeiteten Lösungsvorschläge in geeigneter Form zu exportieren.

Nach einer kleineren Analysephase und einer Beratungsrunde mit unserem Betreuer fiel die Wahl nach der geeigneten Form auf Markdown \cite{markdown}. Der Vorteil von Markdown sahen wir vor allem darin, dass man Markdown schnell und einfach z.B. auf GitHub hochladen kann. Da man als Entwickler tendenziell viel mit GitHub oder ähnlichen Tools/Webseiten arbeitet, empfanden wir diese Umsetzung am  einfachsten und intuitivsten von der Handhabung.

\paragraph*{Implementation Backend}~\\
Die Möglichkeit des Exports wird hauptsächlich durch das Backend bereit gestellt. Dabei wurden die Business-Objekte um eine Serialize-Methode erweitert, welche beschreibt, wie die Objekte im Markdown aussehen sollten bzw. welche Informationen wie (Text, BoldText, ItalicText, Header, etc.) dargestellt werden sollten. 

Im nachfolgenden Listing wird aufgezeigt, wie die Export-Funktion und die Serialize-Methode im Speziellen für das \texttt{BrainstormingFinding} im Backend umgesetzt wurde. Die Umsetzung für die anderen Business-Objekte ist entsprechend gleich aufgebaut. 

\begin{lstlisting}[caption={Serialize-Methode von BrainstormingFinding}, label=markdownBrainstormingFinding]
public class BrainstormingFinding implements MarkdownCascadable {
...
public String getPredecessor() {
    return new Heading(getName(),1).toString() + "\n";
}

...
public String getSuccessor() {
    StringBuilder successor = new StringBuilder();

    successor.append(new BoldText("Basic Information").toString()).append("\n")
    .append(new Text("Description: ").toString())
    .append(getProblemDescription()).append("\n")
            ...
    return successor.toString();
}

...
public String serialize() throws MarkdownSerializationException {

    if (getName().equals("") || getNrOfIdeas() == 0 || getType().equals("") || getBrainstormingTeam().equals("")) {
        throw new MarkdownSerializationException("name is null or description");
    }

    StringBuilder brainsheets = new StringBuilder();

    for (Brainsheet brainsheet: getBrainsheets()){
        brainsheets.append(brainsheet.serialize());
    }
    
    return getPredecessor() + getSuccessor() + brainsheets.toString();
	}
}

\end{lstlisting}

Im \textit{FindingController} kann nun in der entsprechenden Export-Methode nur noch die Serialize-Methode des BrainstormingFindings (Zeile 6) aufgerufen werden.

\begin{lstlisting}[caption={Export-Methode vom FindingController}, label=markdownFindingController]
public Result exportBrainstorming(...){
try {
    BrainstormingFinding finding = service.getFinding(findingIdentifier).get();

    if (finding != null) {
        StringBuilder sb = new StringBuilder().append(finding.serialize());
        return ok(sb.toString());
    } else {
        return badRequest(Json.toJson(new ErrorMessage("Error", "No brainstormingFinding found")));
    }

} catch (InterruptedException | ... e) {
    return internalServerError(Json.toJson(new ErrorMessage("Error", e.getMessage())));
}
}
\end{lstlisting}

\paragraph*{Implementation Frontend}~\\


\subsubsection{Verwendete Bibliotheken im Backend}
Tabelle \ref{tab:verwendete-libraries-play} zeigt die Bibliotheken auf, die im Backend verwendet werden. 
\begin{table}[h]
	\centering
	\begin{tabular}{| l | l | c | l |}
		\hline
		\textbf{Bibliothek} & \textbf{Version} & \textbf{Repository} & \textbf{Lizenz}\\
		\hline
		swagger-play2 & 1.6.0 & \href{https://mvnrepository.com/artifact/io.swagger/swagger-play2_2.12/1.6.0}{mvnrepository.com} & Apache 2.0 \\
		java-jwt & 3.2.0 & \href{https://mvnrepository.com/artifact/com.auth0/java-jwt/3.2.0}{mvnrepository.com} & MIT \\
		mongodb-driver-async & 3.8.0 & \href{https://mvnrepository.com/artifact/org.mongodb/mongodb-driver-async/3.8.0}{mvnrepository.com} & MIT \\
		modelmapper & 2.3.2 & \href{https://mvnrepository.com/artifact/org.modelmapper/modelmapper/2.3.2}{mvnrepository.com} & Apache 2.0 \\
		markdowngenerator & 1.3.1.1 & \href{https://mvnrepository.com/artifact/net.steppschuh.markdowngenerator/markdowngenerator}{mvnrepository.com} & Apache 2.0 \\
		\hline
	\end{tabular}
	\caption[Story-Points]{Verwendete Bibliotheken Backend}
	\label{tab:verwendete-libraries-play}
\end{table}

\subsubsection{Verwendete Bibliotheken im Frontend}
In Tabelle \ref{tab:verwendete-libraries-frontend} sind die verwendeten Libraries und Frameworks des Frontends aufgelistet.
\begin{table}[!h]
	\centering
	\begin{tabular}{| l | l | c | l |}
		\hline
		\textbf{Bibliothek} & \textbf{Version} & \textbf{Repository} & \textbf{Lizenz}\\
		\hline
		CarouselView.FormsPlugin & 5.2.0 & \href{https://github.com/alexrainman/CarouselView}{github.com} & MIT \\
		Microsoft.AppCenter & 1.10.0 & \href{https://visualstudio.microsoft.com/app-center/}{AppCenter} & MIT \\
		NUnit & 3.11.0 & \href{http://nunit.org}{nunit.org} & MIT\\
		NUnit3TestAdapter & 3.11.2 & \href{https://github.com/nunit/docs/wiki/Visual-Studio-Test-Adapter}{github.com} & MIT\\
		Prism.Forms & 7.0.0.396 & \href{https://github.com/PrismLibrary/Prism}{github.com} & MIT \\
		Xamarin.Forms & 3.3.0.912540 & \href{https://docs.microsoft.com/en-us/xamarin/xamarin-forms/}{Microsoft Docs} & MIT \\
		XamlStyler.Console & 3.0.0 & 
		\href{https://github.com/Xavalon/XamlStyler}{github.com} & Apache 2.0\\
		ZXing.Net.Mobile & 2.4.1 & \href{http://github.com/Redth/ZXing.Net.Mobile}{github.com} & Apache 2.0\\
		ZXing.Net.Mobile.Forms & 2.4.1 &
		\href{http://github.com/Redth/ZXing.Net.Mobile}{github.com} & Apache 2.0\\
		\hline
	\end{tabular}
	\caption{Direkt verwendete Bibliotheken Frontend}
	\label{tab:verwendete-libraries-frontend}
\end{table}


\subsubsection{Vergleich Soll/Ist}
%TODO