\subsection{Domainanalyse}

Das Domain-Modell besteht grob aus zwei Teilen: den Benutzern und der Brainstorming Methodik. 

Dabei bilden mehrere Participants ein \textit{BrainstormingTeam}. Diese wird von einem der Participants, dem \textit{Moderator}, gegründet.  Das Team hat die Möglichkeit, ein oder mehrere \textit{BrainstormingFindings} zu erarbeiten. Dies entspricht einer gesamten Durchgang der Methode. Der Moderator erstellt diese und kann zwischen zwei Typen von Findings auswählen. Der Unterschied des \textit{SoftwareFindings} zum \textit{GeneralFinding} liegt darin, dass es im SoftwareFinding eine Auswahl von Software-Patterns gibt, die als Lösungsvorschlag verwendet werden können. Der Moderator kann weiter die Anzahl von Ideen sowie die erste Rundenzeit konfigurieren. Jede weitere Runde wird um eine Minute verlängert.

Das \textit{Brainsheet} entspricht einem physikalischem Blatt, das herumgegeben wird. In der Standardkonfiguration 635 existieren also 6 Sheets (weil 6 Teilnehmer dabei sind).

Eine \textit{Brainwave} ist das Produkt jedes Participants am Ende einer Runde. Es gehört in ein Brainsheet, das jede Runde an den nächsten Participant weitergegeben wird. In der Standardkonfiguration besteht eine Brainwave aus 3 Ideen (6\textbf{3}5).

Die \textit{Idea} ist ein effektiv erarbeiteter Teil einer Brainwave. Im Normalfall ist eine Idee simpler Text (\textit{NoteIdea}), wobei weitere Typen von Ideen (Bild, Weblink und Zeichnung) durch das verwendete Design erdenklich sind. Hierbei ist allerdings zu beachten, dass die \textit{PatternIdea} nur im Software Finding verfügbar ist.
\begin{figure}[h]
	\centering
	\includegraphics[width=1\linewidth]{img/domain-analyse/DomainModell-Methode635}
	\caption{Domain Modell BrainingOutOfBox}
	\label{fig:domainmodell-methode635}
\end{figure}
