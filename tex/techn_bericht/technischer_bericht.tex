\section{Technischer Bericht}

\subsection{Einleitung und Übersicht}
Da die vorliegende Bachelorarbeit auf der Studienarbeit ''Methode 635 als Cross Plattform App mit Xamarin'' \cite{methode635-sa} aufbaut, wird an einzelnen Stellen in diesem technischen Bericht auf die Studienarbeit verwiesen. Auch wird davon ausgegangen, dass dem Leser Begriffe wie ''Methode 635'' oder ''Xamarin'' bereits geläufig sind. Ist dies nicht der Fall, können allfällige Wissenslücken in diesen Bereichen in der erwähnten Studienarbeit nachgelesen werden.

Der technische Bericht selbst besteht neben diesem Kapitel noch aus weiteren Unterkapiteln. 

Dem Leser soll zunächst nochmals ein kurzer Überblick gegeben werden, was am Ende der Studienarbeit erreicht wurde und was der Stand der Entwicklung war, mit welcher diese Arbeit gestartet werden konnte. Wie schon in der Studienarbeit führten wir auch bei der Bachelorarbeit zunächst eine Vorstudie durch, in der wir grundsätzliche Analysen durchführten und Fragen klärten. Dabei ging es vor allem um die Frage, wie Dateien in einer Datenbank gespeichert werden können. All dies ist im Kapitel ''Vorstudie'' niedergeschrieben. Im Kapitel der ''Anforderungsspezifikation'' listen wir alle funktionalen sowie nicht-funktionalen Anforderungen auf, welche neu dazu kamen. Das Domain-Modell findet sich im Kapitel ''Domainanalyse''. Die ''Architektur\-dokumentation'' ist das nächste Kapitel und dokumentiert die logische Architektur, sowie das Deployment. Jegliche Entscheidungen, welche wir in Bezug auf die Architektur getroffen haben, werden im Kapitel ''Architekturentscheide'' begründet. Während der gesamten Umsetzung der Applikation sind natürlich auch Probleme aufgetreten. Diese haben wir im Kapitel ''Herausforderungen'' zusammengetragen. Die eigentliche Implementation haben wir im Kapitel ''Ergebnisse'' dokumentiert. Zum Schluss blicken wir im Kapitel ''Schlussfolgerungen'' nochmals kritisch auf unser Projekt zurück und bewerten unsere Arbeit und geben einen Ausblick, wie man die Applikation noch erweitern könnte.

\subsection{Ergebnis der Studienarbeit}
Aus der vorherigen Studienarbeit ist eine lauffähige Applikation hervorgegangen, welche für verschiedene Brainstormings verwendbar ist. Die einzelnen Durchführungen können zudem in Punkto Teilnehmeranzahl, Rundenanzahl und Rundenzeit separat konfiguriert werden. Ein durchgeführter Test ergab allerdings, dass die Applikation in Sachen Stabilität und Benutzerführung noch Potenzial für Verbesserungen aufwies.

Wegen der beschränken Zeit von 14 Wochen für die Studienarbeit konnte zudem die Eingabe von neuen Ideen nur via Text umgesetzt werden (nicht wie zunächst angedacht auch mit Bilder, Skizzen, etc.).

Das Ziel der Studienarbeit war es einen lauffähigen Prototypen zu entwickeln. Unser Fokus stand daher primär auf der Umsetzung jenes Prototypen. Kombiniert mit der beschränkten Zeit führte dies dazu, dass der Code gegen Ende des Projektes immer umfangreicher und unübersichtlicher wurde. Um dem in einer allfälligen Folgearbeit entgegenzuwirken, entwickelten wir noch in der Studienarbeit entsprechende Konzepte und Ideen. 

Weiterführende Informationen zur Ergebnisbewertung der Studienarbeit kann im gleichnamigen Kapitel jener Arbeit \cite{methode635-sa} nachgelesen werden.

\subsection{Vorstudie}
Dieser Abschnitt dokumentiert die Vorarbeiten in verschiedenen Bereichen, die für das Abwickeln dieses Projektes relevant sein können. Dabei werden die entsprechenden Themen analysiert und es wird abgewägt, inwiefern diese für den Projekterfolg von Nutzen sein können. Stellt sich heraus, dass eines der analysierten Themen sinnvoll und machbar ist, wird es in das Projekt integriert.

\subsubsection{Solution Strategy}\label{sec:sol-strategy}

In der von Gernot Starke entwickelten Vorlage zur Software Architektur Dokumentation arc42 \cite{arc-42} sowie in seinem Buch \textit{Effektive Softwarearchitekturen} \cite{eswa}, befindet sich ein für uns besonders interessanten Abschnitt namens \textit{Solution Strategy}. Darin ist ein Vorschlag angeboten, wie man in der Synthese der Software Architektur die Ansätze zur Lösungsfindung erarbeitet und dokumentiert. Dieser Vorschlag besteht aus mehreren Tipps zum Inhalt, zur Darstellung und zur Entwicklung dieser Ansätze. Als Beispiel beinhaltet dieser Vorschlag folgenden Tipp: ``Erkläre die Lösungsstrategie so kompakt wie möglich (z.B. als Liste von Schlüsselwörtern)``. Da sich dieses Projekt intensiv mit Lösungsfindungsmethoden auseinandersetzt, könnte die Solution Strategy nützlichen Input bieten, wie der Lösungfindungsprozess in unserer Applikation gestaltet werden könnte. 

Eine mögliche Erweiterung unserer Applikation wäre, ein zusätzliches Modul für eine Lösungsfindungsvariante anzubieten. Dabei käme beim Erstellen eines Brainstormingfindings die Auswahl zwischen zwei Typen von Lösungsfindungen: ``Software Architektur Lösung`` und ``Generelle Lösung``. Bei der Software Lösung würden dann die Tipps von arc42 eingearbeitet sein, wobei eine Abbildung gemäss Tabelle \ref{tab:arc42-mapping} vom Kontext des arc42-Templates in den Applikationskontext denkbar wäre.

\renewcommand{\arraystretch}{1.7}
\begin{table}
	\centering
	\begin{tabular}{|p{6cm}|p{6cm}|}
		\hline
		\textbf{arc42 \& eswa\tablefootnote{Buch Effektive Softwarearchitekturen}-Kontext} & \textbf{Applikationskontext}\\
		\hline
		Lösungsansatz finden & Aufgabe von Benutzer, Kreativität ist von Benutzer gefragt\\
		\hline
		Solution Strategy so genau wie möglich erklären (z.B. als Liste von Keywords) & Liste von Keywords mittels Textinput \\
		\hline
		Lösungsansatz als Tabelle beschreiben & Tabellarische Darstellung mithilfe der Skizzenfunktion\\
		\hline
		Lösungsansatz im Kontext der Qualtätsattribute beschreiben & Bezugnahme auf Qualitätsattribute mittels Textinput, oder Auswahlliste mit definierten Qualitätsattribute anbieten\\
		\hline
		Konzepte, Views oder Code referenzieren & Liste von definierten (Microservices-API-)Patterns\cite{microservices-api} als Input anbieten (z.B. \textit{Frontend Integration}, mit Icon und kurzem Beschrieb des Patterns)\\
		\hline
		Lösungsansatz inkrementell und iterativ wachsen lassen & Durch rundenbasiertes Brainwriting gegeben. \\
		\hline
		Lösungsansatz rechtfertigen, vergleichen, entscheiden und begründen & Teil der nachträglichen Diskussion, nicht Teil der Applikation (Wertung und definitive Entscheide sollten nicht in Brainwriting einfliessen), Exportfunktion \\
		\hline
	\end{tabular}
	\caption{Abbildung der Solution Strategy gemäss arc42 und eswa auf Brainstorming Applikation}
	\label{tab:arc42-mapping}
\end{table}
\paragraph{Fazit}~\\
Durch die verschiedenen geplanten Erweiterungen bezüglich den Input-Varianten (siehe Use Case 8 im Kapitel \ref{sec:functional-requirements} Funktionale Anforderungen) lässt sich diese Vorlage sinnvoll in die Applikation integrieren. Dies erfordert allerdings eine erweiterbare Grundlage, um das erwähnte Modul (``Software Architektur Lösung``) zu implementieren. Durch eine saubere Abstraktion sollte die Architektur der Applikation aber auch andere Lösungstypen unterstützen.
\newpage

\subsubsection{Dateien in Datenbank speichern}
\label{seq:save_file_in_db}
Da der wesentliche Bestandteil dieser Bachelorarbeit darin besteht, die bestehende Applikation um verschiedene Medien wie Bilder, Skizzen, Videos oder Links etc. zu erweitern, mussten wir uns zuerst klar werden, wie wir dies erreichen könnten.

Da wir uns während der Studienarbeit \cite{methode635-sa} für MongoDB als No-SQL Datenbank entschieden haben, sahen wir es als sinnvoll an, wieder auf diese Technologie zu setzen, zumal wir auch gute Erfahrungen damit gemacht hatten.

Mit GridFS \cite{gridfs-mongodb}\cite{gridfs-mongodb-async-driver} fanden wir eine Spezifikation, welche es ermöglicht, verschiedenste Dateien in eine MongoDB zu speichern oder von dort wieder herzustellen. Einer der Vorteile von GridFS liegt darin, dass auch grosse Dateien (GB an Daten) in der Datenbank abgelegt werden können. Dabei zerlegt GridFS eine solche Datei in kleinere Teile sogenannte Chunks und speichert jedes dieser Chunks in einem eigenen Document ab \cite{gridfs-mongodb}.

GridFS kann daher nicht nur jegliche Art von Dateien speichern, welche die Standard-Documentgrösse von 16MB übersteigen, sondern kann diese auch wieder zur Verfügung stellen ohne dass die komplette Datei in den Memory geladen werden muss. GridFS bietet somit die Möglichkeit von Streaming \cite{gridfs-chunking}.

Ausserdem limitiert GridFS nicht die Anzahl an Dateien, welche gespeichert werden können, wie es teilweise bei Filesystemen der Fall ist.

Zudem können mit GridFS auch nur Teile einer Datei zur Verfügung gestellt werden. So kann z.B. bei einem Video mittels Range Queries bis zur Mitte ``gesprungen`` werden ohne das die erste Hälfte (herunter)geladen werden muss \cite{gridfs-mongodb}.

GridFS ist laut \href{https://docs.mongodb.com/manual/core/gridfs/}{docs.mongodb.com} allerdings nicht geeignet, falls man den Inhalt einer Datei atomar updaten möchte oder muss. In solch einem Fall muss man die aktualisierte Datei erneut in die Datenbank speichern und als neue Version ablegen. Des Weiteren sollten keine Dateien, welche kleiner als 16MB sind als einzelne Chunks gespeichert werden. Stattdessen wird empfohlen, diese wie gewohnt in einem Document abzuspeichern.

Die Implementation von GridFS wurde unter der MIT Lizenz veröffentlicht.

\paragraph{Fazit}~\\
Mit GridFS steht uns eine Spezifikation zur Verfügung, welche nicht nur alle unsere Bedürfnisse abdeckt sondern auch die Möglichkeit von Streaming bietet, was der Performance unserer gesamten Applikation sicherlich zugute kommen könnte. Der ausschlaggebende Punkt warum wir uns zum Schluss aber für GridFS entschieden haben, war die Tatsache, dass diese Spezifikation schon in unserem verwendeten Treiber implementiert ist. So konnten wir mit sehr wenig Aufwand GridFS für unser Projekt verwenden.
\newpage

\subsubsection{Code Review durch Senior Entwickler}
Der Quellcode der gesamten Applikation, welcher im Umfang der Studienarbeit entstand, wurde durch Silvan Gehrig, ein wissenschaftlicher Mitarbeiter des Instituts für Software, geprüft. Dadurch konnten einige Mängel und Punkte zur Verbesserung identifiziert werden. Die wichtigsten Kritikpunkte sind nachfolgend aufgelistet und erläutert.

Der komplette Bericht von Silvan Gehrig kann im Anhang \ref{sec:code-review} nachgelesen werden.

\paragraph{Server}~\\
Auf dem Backend wurden folgende Punkte bemängelt:
\begin{basedescript}{
		\desclabelstyle{\multilinelabel}
		\desclabelwidth{4.5cm}
		\setlength{\itemsep}{5ex}}
		\item[Unsauberes Layering] Controller greift direkt auf Datenbank zu.
		\item[Keine DTOs vorhanden] Controller arbeiten teilweise direkt mit JSON Nodes anstatt mit DTOs.
		
		\item[Technologie zielgerichtet einsetzen] Zum Teil zu lange Controller-Klassen, des Weiteren existieren keine Unit Tests
		
\end{basedescript}
\paragraph{Client}~\\
Clientseitig gilt es folgende Punkte zu verbessern:
\begin{basedescript}{
	\desclabelstyle{\multilinelabel}
	\desclabelwidth{4.5cm}
	\setlength{\itemsep}{5ex}}
\item[UI-Strings nicht in Resource Files] Besser als die UI-Strings im ViewModel zu definieren ist das Verwenden von Resource Manifests.
\item[Layering] Business Services sind im ViewModel implementiert, anstelle von separaten Business Logik Klassen. Zum Teil wird direkt vom ViewModel auf den DAL zugegriffen.
\item[Technologie zielgerichtet einsetzen] Separate Projekte pro Layer, Unit Tests fehlen.
\end{basedescript}
\vspace{0.5cm}
\paragraph{Fazit}~\\
Da zu Beginn der Construction Phase (siehe \ref{subsec:timeline}) zwei Sprints für die Überarbeitung und Verbesserung des bestehenden Codes geplant sind, werden diese und weitere Punkte in dieser Zeit verbessert. Dadurch erhoffen wir uns eine ausgereiftere Plattform, für die sich zukünftige Features effizient und einfach entwickeln lassen. 
\newpage

\subsection{Anforderungsspezifikation}

\subsubsection{Funktionale Anforderungen}

\paragraph{Brief Use-Cases}


\paragraph{Fully-Dressed Use-Cases}


\paragraph{Abuse-Cases}
\paragraph{Sequenzdiagramm}

\subsubsection{Nicht-Funktionale Anforderungen}
Wie schon in unserer Studienarbeit halten wir uns auch hier wieder an die Standards ISO 9126\cite{ISO9126} bzw. dessen Nachfolger ISO 25010\cite{ISO9126_ISO25010}. Beide ISO-Normen sind sich sehr ähnlich und liefern eine gute Checkliste für jegliche Art von Systemanforderungen.

\begin{figure}[h]
	\centering
	\includegraphics[width=1\linewidth]{img/anforderungen/quality}
	\caption[Anforderungskategorien nach ISO 25010]{Anforderungskategorien nach  ISO 25010}\cite{ISO25010_Bild}
	\label{fig:ISO 25010}
\end{figure}

Im Gegensatz zur Studienarbeit wollen wir uns bei dieser Arbeit aber vor allem auf Faktoren wie Erweiterbarkeit und Modularität konzentrieren. Auch sollen Faktoren wie Zeitverhalten und Ästhetik stärker in den Vordergrund rücken. 

Die bekannten nicht-funktionale Anforderungen aus der Studienarbeit bleiben allerdings weiter bestehen. Um genaue und erfüllbare nicht-funktionale Anforderungen zu definieren, müssen die SMART-Kriterien \cite{SMART} erfüllt sein.

\begin{center}
    \begin{tabular}{ | p{6cm} | p{2.5cm} | p{2.5cm} | p{2.5cm} |}
    	\hline
    Kriterium & Minimum & Optimal & Übertroffen \\ 
    	\hline
    \textbf{Zeitverhalten} \newline Zeitkritische Kommunikation (Abgabe von Ideen/Brainsheets) zwischen Server und App beträgt: & 2 Sekunden & 1 Sekunden & weniger als 1 Sekunde \\
    	\hline
    \textbf{Erweiterbarkeit} \newline Die Anzahl an neuen Modularten (Problem-Arten), welche mit der bestehenden Architektur als umsetzbar gelten, wird als ... angesehen: & zu wenig & ausreichend & unbegrenzt \\
    	\hline
    \textbf{Modularität} \newline Die App ist innert ... Tagen um ein neues Modul (Problem-Art) erweitert: & 5 Tage & 2 Tage & weniger als 1 Tag \\
    	\hline
    \textbf{Ästhetik} \newline Die Anziehungskraft gegenüber dem Endnutzer wird als ... charakterisiert: & gering & in Ordnung & süchtig \\
    	\hline
    \textbf{Ausgereiftheit} \newline Der Grad der Ausgereiftheit oder Reife wird als ... angesehen: & ungenügend für den produktiven Einsatz (Prototypen-Stadium) & genügend für den produktiven Einsatz, kann aber immer noch in Fehlerzustände gelangen & kaum Fehlerzustände und somit keine Abstürze der App  \\
    	\hline
    \end{tabular}
\end{center}

\newpage

\subsection{Domainanalyse}

Das Domain-Modell besteht grob aus zwei Teilen: den Benutzern und der Brainstorming Methodik. 

Dabei bilden mehrere \textit{Participants} ein \textit{BrainstormingTeam}. Diese wird von einem der \textit{Participants}, dem \textit{Moderator}, gegründet.  Das Team hat die Möglichkeit, ein oder mehrere \textit{BrainstormingFindings} zu erarbeiten. Dies entspricht einem gesamten Durchgang der Methode. Der \textit{Moderator }erstellt diese und hat die Möglichkeit, die Anzahl von Ideen sowie die erste Rundenzeit zu konfigurieren. Jede weitere Runde wird um eine Minute verlängert.

Das \textit{Brainsheet} entspricht einem physikalischem Blatt, das herumgegeben wird. In der Standardkonfiguration 635 existieren also 6 Sheets (weil 6 Teilnehmer dabei sind).

Eine \textit{Brainwave} ist das Produkt jedes \textit{Participants }am Ende einer Runde. Es gehört in ein \textit{Brainsheet}, das jede Runde an den nächsten \textit{Participant }weitergegeben wird. In der Standardkonfiguration besteht eine \textit{Brainwave }aus 3 Ideen (6\textbf{3}5).

Die \textit{Idea} ist ein effektiv erarbeiteter Teil einer \textit{Brainwave}. Im Normalfall ist eine Idee simpler Text (\textit{NoteIdea}), wobei weitere Typen von Ideen (Bild, Weblink und Zeichnung) durch das verwendete Design erdenklich sind. Speziell erwähnenswert ist der Umstand, dass im Gegensatz zur Studienarbeit, neu die Ideentypen \textit{SketchIdea }und \textit{PatternIdea }angedacht sind.

\begin{figure}[h]
	\centering
	\includegraphics[width=1\linewidth]{img/domain-analyse/DomainModell-Methode635}
	\caption{Domain Modell BrainingOutOfBox}
	\label{fig:domainmodell-methode635}
\end{figure}


\newpage

\subsection{Architekturdokumentation}
\label{architektur}
%TODO Architektur anpassen

In diesem Kapitel gehen wir detailliert auf die Architektur und das Deployment unseres Projektes ein. Es ist allerdings zu vermerken, dass dieses Kapitel grösstenteils aus der Studienarbeit \cite{methode635-sa} hervorgegangen ist. Es soll hier dennnoch der Vollständigkeit halber aufgeführt werden.

\subsubsection{Logische Architektur}
Wir teilen die Architektur des gesamten Systems in drei Schichten auf. In der Abbildung \ref{fig:architektur-methode635} sind diese als Presentation-, Businesslogic- und Persistence-Schicht zu erkennen.

\begin{figure}[h]
	\centering
	\includegraphics[width=1\linewidth]{img/architektur/CD_Methode635}
	\caption{Logische Architektur BrainingOutOfBox}
	\label{fig:architektur-methode635}
\end{figure}

Die Präsentationsschicht ist die Schicht über die der Benutzer mit der Xamarin App kommuniziert. Konkreter gesagt, umfasst diese die verschiedenen View-Komponenten, welche für das Aussehen der App verantwortlich sind. Jegliche Interaktionen über die Oberfläche werden anschliessend in der Schicht der Businesslogic weiter verarbeitet. In dieser Schicht haben wir wieder unsere Xamarin App, welche selbst Logik-Komponenten wie die Timing-Komponente oder weitere App spezifische Logik-Komponenten enthält.

Die AppLogic-Komponente ist für das korrekte Verarbeiten und Weiterleiten der Eingaben an das PlayFramework verantwortlich.

Die AppTiming-Komponente ist zuständig für das Zeitmanagement während dem Brainstorming. Diese Komponente überwacht daher die noch verbleibende Zeit.

Die AppStateMachine-Komponente ist für den Runden-Wechsel-Mechanismus zuständig. 

Um die Inhalte des Brainstormings sowie die Benutzerdaten an das Backend zu übermitteln, gibt es die AppDataAccess-Komponente. Im Gegensatz zur DataAccess-Komponente auf dem Backend verarbeitet diese Komponente JSON-Files. 

Auf der anderen Seite haben wir das PlayFramework, welches wiederum Access-Komponenten, Routing-Komponenten, eine Timing-Komponente, Logik-Komponenten und eine DataAccess-Komponente enthält.

Die Timing-Komponente und die Logic-Komponente haben die selben Aufgaben wie ihre Gegenstücke in der Xamarin App. Auch hier verwalten diese das Zeitmanagement während den einzelnen Runden und stellen sicher, dass nach Ablauf der Zeit oder sobald alle \textit{Brainsheets} abgegeben wurden, eine neue Runde beginnt. Des Weitern ist die Logic-Komponente zum Beispiel verantwortlich, dass ein \textit{Participant} einer Gruppe nicht zweimal beitreten kann oder diese verlassen kann, wenn er sie schon einmal verlassen hat. 

Die DataAccess-Komponente stellt sicher, dass jegliche Daten korrekt geladen oder gespeichert werden.

Die Persistence-Schicht ist für alle Datenzugriffe zuständig. 

Konkret steht uns je ein DataStore für die \textit{BrainstormingFindings}, für die \textit{BrainstormingTeams} und für die \textit{Participants} zur Verfügung.

\paragraph*{Komponenten}~\\

Nachfolgend sind nochmals alle Komponenten aufgelistet und kurz beschrieben. Für eine ausführlichere Beschreibung ist der Text oberhalb zu lesen.

\begin{description}[leftmargin=!,labelwidth=\widthof{\bfseries AppStateMachineComponent}]
	
	\item[ViewComponent] Die View-Komponenten der Xamarin App sind für das korrekte Anzeigen der Informationen verantwortlich. Sie definieren das Aussehen der Applikation.
	
	\item[AppTimingComponent] Die Xamarin App hält in der logischen Schicht eine Timing-Komponente, welche dafür sorgt, dass ein 
	\textit{BrainstormingFinding} nach Ablauf der Zeit abgesendet wird.
	
	\item[AppStateMachineComponent] Die Komponente für das Evaluieren des Zustandes des Brainstormings.
	
	\item[AppLogicComponent] Die Logik-Komponente der Xamarin App regelt weitere Logik, wie z.B. das Hinzufügen einer Idee.
	
	\item[AppDataAccessComponent] Die Komponente für das Serialisieren und Deserialisieren von JSON Dateien für das und vom Backend.
	
	\item[AccessComponent] Die Access-Komponente auf dem PlayFramework regelt den Zugriff mittels JWT-Token. JWT-Tokens werden bei erfolgreichem Login an den Benutzer der App gesendet. 
	
	\item[Routing] Die Routing-Komponente sorgt anhand der URL für das Aufrufen der korrekten Funktion.
	
	\item[TimingComponent] Wie die Xamarin App hält auch das PlayFramework eine Timing-Komponente, um den Zustand der Zeit verwalten zu können.
	
	\item[LogicComponent] In der Logik-Komponente werden die eigentlichen Funktionen geschrieben. Hier ist auch die Logik für den Austausch der Blätter untergebracht.
	
	\item[DataAccessComponent] Die DataAccess-Komponente stellt das Bindeglied zwischen dem PlayFramework und dem DataStore dar. Es ermöglicht erst den Zugriff auf die gespeicherten Daten.
	
	\item[PersistenceComponent] Die Persistence-Komponente regelt das korrekte und dauerhafte Speichern in die einzelnen DataStores.
\end{description}

\subsubsection{Deployment}
Wie in der Abbildung \ref{fig:deployment-methode635} zu sehen ist, besteht unser System aus zwei physikalischen Geräten. Das ist zum einen der Client und zum anderen der BackendNode. Diese beinhalten jeweils sogenannte \textit{DeploymentUnits} (DU). 

\begin{figure}[h]
	\centering
	\includegraphics[width=1\linewidth]{img/deployment/DD_Methode635}
	\caption{Deploymentdiagramm BrainingOutOfBox}
	\label{fig:deployment-methode635}
\end{figure}

Beim Client handelt es sich um das Smartphone des jeweiligen Benutzers. Auf seinem Smartphone läuft die Xamarin App, welche wiederum die appPresentationLayerDU und die appBusinessLogicLayerDU hält.

Der BackendNode ist ein Ubuntu 18.04 auf dem ein Java Runtime Environment (JRE) installiert ist. Innerhalb der JRE läuft das PlayFramework, in dem wiederum die businessLogicLayerDU läuft.

Zudem ist auf dem BackendNode ein MongoDB Service installiert, welche  die persistenceLogicLayerDU beinhaltet.

\paragraph*{Komponenten}
Nachfolgend sind nochmals alle DeploymentUnits aufgelistet und kurz beschrieben.
\begin{description}[leftmargin=!,labelwidth=\widthof{\bfseries appBusinessLogicLayerDU}]
	\item[businessLogicLayerDU] Die businessLogicLayerDU enthält alle Komponenten, welche in Abbildung \ref{fig:architektur-methode635} in der Businesslogic-Schicht im PlayFramework eingezeichnet sind.
	\item[persistenceLogicLayerDU] Die persistenceLogicLayerDU beinhaltet alle Komponenten der Persistence-Schicht, welche in Abbildung \ref{fig:architektur-methode635} zu sehen ist.
	\item[appPresentationLayerDU] Die appPresentationLayerDU enthält alle Komponenten, welche in Abbildung \ref{fig:architektur-methode635} in der Präsentationsschicht liegen.
	\item[appBusinessLogicLayerDU] Die appBusinessLogicLayerDU enthält alle Komponenten, welche in Abbildung \ref{fig:architektur-methode635} in der Businesslogic-Schicht in der Xamarin App gezeichnet sind.
\end{description}
\newpage

\subsection{Architekturentscheide}
\label{subsec:architecture-decisions}
Wesentliche Entscheide, welche wir während dem Projekt getroffen haben, sind hier detailliert begründet. Auch Gedanken oder Ideen, welche wir während der Analysephase hatten, dann aber wieder verworfen haben, sind hier mit Begründung aufgeschrieben.

\subsubsection{Domain-Design-Entscheid}
Schon sehr früh im Projekt kam die Idee auf, die BrainstormingFindings in verschiedene Arten zu unterteilen. Die Rede war von ``Software Architektur Lösung`` (SoftwareFinding) und ``Generelle Lösung`` (GeneralFinding). Siehe dazu Kapitel \ref{sec:sol-strategy}. Damit wollten wir erreichen, dass nicht jeder Ideentyp für jedes BrainstormingFinding zur Verfügung steht, zumal es auch nicht unbedingt für jede Kombination Sinn ergibt, diesen Ideentypen dafür zu verwenden. Zum Beispiel ist es wenig sinnvoll die PatternIdea bei einem GeneralFinding zu nutzen. Dies würde eher bei einem SoftwareFinding zum Einsatz kommen.

\begin{figure}[h]
	\centering
	\includegraphics[width=1\linewidth]{img/domain-analyse/DomainModell-Methode635-Entwurf}
	\caption{Erster Entwurf vom Domain Modell BrainingOutOfBox}
	\label{fig:domainmodell-methode635-entwurf}
\end{figure}

Doch dabei kommt das Bedenken auf, was wenn es noch andere Patterns als nur Software-Patterns gibt, welche man zur Verfügung stellen will. Dann müssten noch mehr solcher logischen Zusammenhänge beachtet werden, was die Erweiterbarkeit der Applikation mit zunehmenden BrainstormingFinding Arten erschwert.

Je länger wir untereinander aber auch mit unserem Betreuer, Herr Zimmermann oder Silvan Gehrig darüber diskutierten, desto mehr kamen wir wieder auf unser ursprüngliches Domain-Design zurück, bei welchem es nur ein BrainstormingFinding gibt. 

Dabei ist die Idee, alle Ideentypen zwar anzubieten aber die Auswahl des passenden Ideentypen komplett dem Endnutzer zu überlassen. Dies birgt allerdings die Gefahr, die Benutzung der Applikation durch eine zu grosse Auswahl an Ideentypen zu verschlechtern. Dieses Problem kann aber durch eine durchdachte Aufteilung in der Benutzeroberfläche eingedämmt werden. Unserer Meinung nach ist mit dem verwendeten Domainmodell (siehe Abbildung \ref{fig:domainmodell-methode635}) die Erweiterbarkeit für zukünftige Arbeiten eher gegeben, da weniger Stellen im Code bearbeitet werden müssen. 

Aus diesem Grund haben wir uns für das ursprüngliche Domainmodell mit nur einem BrainstormingFinding entschieden. Die Tipps von arc42 bzw. der Solution Strategy konnten dennoch gemäss Tabelle \ref{tab:arc42-mapping} integriert werden.

\subsubsection{Architekturentscheide der Studienarbeit}
Nachfolgend listen wir die Entscheide auf, welche in der vorangegangenen Studienarbeit in Bezug auf die eingesetzten Technologien getroffen wurden. Es soll dem Leser nochmals  verständlich machen, warum genau diese Technologien für diese Arbeit gewählt wurden.
Die folgenden Kapitel sind daher der Studienarbeit \cite{methode635-sa} wortgetreu entnommen.

\paragraph*{Xamarin.Forms oder Xamarin native}\label{subsubsec:forms-vs-native}~\\
Für diesen Entscheid galt es zu evaluieren, welche User Controls für unsere Applikation die exotischsten sind. Dies, weil Xamarin.Forms eine Menge an Standard-Controls anbietet, die vom Framework selber in das jeweilige Betriebssystem konvertiert werden. Sind alle vorgesehenen Benutzerelemente in Forms enthalten, sparen wir uns die Zeit, betriebssystemspezifische Elemente zu entwickeln. 

Für unser Projekt haben wir folgende Benutzerelemente als exotisch oder kritisch definiert:
\begin{itemize}
	\item Canvas Control für Zeichnen einer Idee
	\item Camera Funktion für das Erkennen von Quick Response-Codes (QR-Codes)
	\item Verarbeitung und Generierung von QR-Codes
\end{itemize}

Nach einer Recherche stellte sich heraus, dass sich ein Canvas View von Google namens SkiaSharp eignet. Darauf lässt sich gemäss der Dokumentation zeichnen sowie definierte Formen einfügen. Dies könnte auch für eine Erweiterung spannend sein, in der Patterns in UML als Vorlage angeboten werden können.

Für die Kamera-Funktionalität steht ein NuGet-Packet (Xam.Media.Plugin) bereit, das uns diese Arbeit abnehmen wird.

Das Generieren und Lesen der QR-Codes ist an sich kein Problem von Xamarin.Forms, denn grundsätzlich müssen die von der Kamera generierten Files eingelesen und ins entsprechende QR-Code-Tool eingefügt werden. Hierfür eignet sich das NuGet ZXing.Net. 

Es stellte sich relativ rasch heraus, dass die gewünschten Funktionalitäten in Xamarin.Forms in ausreichender Qualität enthalten sind und uns das individuelle Entwickeln dadurch abgenommen wird.\\
\\


\paragraph*{Backend-Technologie}~\\

Neben den Vorteilen, wie der schlanken und zustandslosen Architektur des Play\-Frameworks, dem asynchronen und nicht-blockierenden Verhalten und den vielen unterstützten Bibliotheken, haben wir uns hauptsächlich dafür entschieden, weil wir in anderen Projekten schon sehr gute Erfahrungen mit dem PlayFramework gemacht haben.

Ein weiterer Grund bestand darin, dass das PlayFramework nicht nur in Scala sondern auch in Java geschrieben ist. Mit Java kennen wir uns beide gut aus und mussten uns so keine neue Programmiersprache aneignen.

Da wir uns schon relativ früh für eine MongoDB als Datenbanksystem entschieden hatten, fiel die Wahl erst recht auf das PlayFramework, als wir einen asynchronen MongoDB-Treiber für Java gefunden hatten.

\paragraph*{MongoDB als Datenbanksystem}~\\

MongoDB (abgeleitet von humongous) ist eine dokumentorientierte, einfache, dynamische und skalierbare NoSQL Datenbank, welche von der MongoDB Inc. entwickelt wird. 

Die Basis für das Speichern von Informationen bilden die sogenannten Documents. Die Datenobjekte werden in separaten Documents innerhalb einer Collection (anders als bei traditionellen relationalen Datenbanken in Spalten und Zeilen) gespeichert. Mit den Documents können zudem  hierarchische Strukturen und Relationen sehr einfach gespeichert werden.

Der Vorteil einer MongoDB Datenbank liegt darin, dass zusammengehörige Informationen gemeinsam in einem Document gespeichert werden. Dies ermöglicht einen schnellen Zugriff auf die Daten mittels der MongoDB Query Language. Da MongoDB zudem ohne Schemas auskommt, ist es nicht nötig, die Datenbank offline zu nehmen, wenn man ein neues Feld einfügen möchte.

Weitere Vorteile sind laut \href{DZone.com}{DZone.com} die hohe Performance, Verfügbarkeit (durch Replikas) und Skalierung (durch Sharding). Da alle Informationen zusammen in einem Document gespeichert sind, sind auch keine Joins zu anderen Tabellen notwendig. Auch unterstützt MongoDB Funktionen für die Speicherung von Geoinformationen.

Der Hauptgrund warum wir uns für MongoDB als Datenbanksystem entschieden haben, war die Tatsache, dass alle zusammengehörigen Informationen in einem Document abgelegt werden können. Auch die Möglichkeit hierarchische Strukturen (bei uns die \textit{Brainsheets}, \textit{Brainwaves} und \textit{Ideas}) einfach zu Speichern, hat uns viel Zeit erspart.

Die Mächtigkeit von relationalen Datenbanken im Bereich einer Auswertung, ist in unserem Projekt nicht notwendig. Daher haben wir auch von Beginn an auf das Paradigma der dokumentorientieren Datenbanken gesetzt. Grund warum wir uns schlussendlich für MongoDB und nicht für eine andere dokumentorientierte Datenbank entschieden haben ist, dass MongoDB Thema während dem Studium ist und wir schon einige Erfahrung damit hatten.
\\
\\

\paragraph*{Runden-Ende-Entscheid}~\\

Der Entscheid, wie eine Runde zu enden hat bzw. was passiert, wenn die Zeit in der laufenden Runde abläuft, der Participant aber noch nicht alle seine Ideen in die Brainwave gespeichert hat, hatten wir schon relativ früh in der Studienarbeit entschieden aber nirgends dokumentiert. Dies soll hier nachgeholt und begründet werden.

Grundsätzlich haben wir uns in solch einem Fall dazu entschlossen, lediglich jene Ideen in der Datenbank zu persistieren, welche auch in der Brainwave gespeichert sind. Im Umkehrschluss bedeutet das, dass alle Ideen, welche zu diesem Zeitpunkt nicht in der Brainwave gespeichert sind (betrifft nur die letzte erfasste Idee), verloren gehen. 

Der Grund warum wir uns dafür entschieden haben liegt einerseits darin, dass dies die einfachste Lösung für die geschilderte Situation darstellt. Da dies zudem nur eine Idee betreffen kann, kann auch nur maximal diese Idee 'verloren' gehen. 

Zudem wird bei der originalen Version mit Papier gleich verfahren. Auch dort wird unabhängig, ob ein Participant alle Ideen ausfüllen konnte oder nicht, das Blatt weitergereicht. 
\newpage

\subsection{Herausforderungen}
\label{subsec:challenges}
Hier sind besonders erwähnenswerte Herausforderungen und Hürden beschrieben, die im Verlaufe des Projektes aufkamen. Dies soll anderen Software Ingenieuren oder Interessierten helfen, aus unseren Schwierigkeiten zu lernen. 


\subsubsection{Navigationsprobleme mit Prism Forms}\label{subsub:navigation}

Die grösste Herausforderung im Frontend war die Navigation mit Prism Forms und Xamarin Forms. Ein zentraler Unterschied liegt in der absolution und relativen Navigation. Bei der absoluten Navigation wird der gesamte Navigationsstack bei jedem Aufruf neu aufgebaut, was zur Folge hat, dass alle beteiligten ViewModels neu erstellt (Konstruktoraufruf) werden und darauf navigiert wird (Aufruf von \texttt{OnNavigatedTo} und \texttt{OnNavigatedFrom}). Dies im Unterschied zur relativen Navigation, wo jeweils die zu navigierende Page auf den Stack gelegt wird und nur die Methoden auf dem zuständigen ViewModel aufgerufen werden. 

Das Problem war nun, dass wir uns im Kontext einer \texttt{TabbedPage} bewegen und die Navigationsschritte bei einem Klick auf einem Element in einem Tab auf ein anderes Tab wechseln müssen. Leider funktioniert in diesem Zusammenhang nur die absolute Navigation, weil wir innerhalb einer Page mit den verschiedenen Tabs navigieren wollen und die Navigation von Prism Forms dazu implementiert wurde, zwischen Pages zu navigieren. Dies hat zur Folge, dass jeder Tabwechsel mit absoluter Navigation entwickelt werden musste. Der gesamte Code muss also berücksichtigen, dass die ViewModels mehrmals initialisiert werden und somit Instanzen überschrieben und verändert werden können. Dies ist ein erheblicher Mehraufwand, der immer wieder zu Fehlverhalten und viel zusätzlichem Code führte. 

Im späteren Projektverlauf wurde festgestellt, dass der Navigationservice von Prism Forms um eine Methode namens \texttt{SelectTab()} erweitert wurde \cite{prism-selecttab}, was die beschriebene Problematik lösen könnte. Jedoch ist diese Extensionmethode noch nicht im stabilen Release von Prism Forms und daher noch nicht nutzbar. In einer Folgearbeit ist es sehr empfehlenswert, diese Funktionalität zu testen und zu verwenden. 


\newpage

\subsection{Ergebnisse}
Das Kapitel der Ergebnisse befasst sich mit der konkreten Umsetzung der gesamten Applikation und dokumentiert die grundlegenden Ergebnisse oder Resultate, welche aus dieser Arbeit hervorgegangen sind. 

Auch hier soll nochmals erwähnt sein, dass diese Arbeit auf der Studienarbeit \cite{methode635-sa} aufbaut und grundlegende Informationen dort nachzulesen sind. 

In diesem Kapitel gehen wir zunächst auf die Änderungen während der Refactoring-Phase ein und schildern was dies für die Wartung und Erweiterung des Codes bedeutet. 

Anschliessend sind weitere Features dokumentiert, welche im Verlauf dieser Arbeit an der Applikation durchgeführt wurden.

Das Kapitel wird durch eine Gegenüberstellung der erreichten und geplanten Arbeit abgeschlossen.

Zur besseren Übersicht wurde der Code vereinzelt gekürzt. Dies ist durch 3 Punkte (...) gekennzeichnet.

\subsubsection{Refactoring zu Beginn der Arbeit}
Wie aus dem Projektplan (siehe Abbildung \ref{fig:projekt-plan}) zu entnehmen ist, haben wir uns zu Beginn unserer Arbeit dazu entschieden, zwei Sprints für das Refactoring unsere Applikation zu nutzen. Die Entscheidung überhaupt ein Refactoring durchzuführen entstand aus der Tatsache, dass wir schon gegen Ende der Studienarbeit gemerkt hatten, dass bei einer allfälligen Weiterführung der Arbeit ein Refactoring von grossem Wert sein wird. Auch haben wir diesbezüglich schon während der Studienarbeit einzelne Vorschläge für ein Refactoring beschrieben.


Das Ziel der Refactoring-Phase war es daher, den Code welcher aus der Studienarbeit hervorgegangen ist, zu prüfen und zu verbessern. Dies sollte uns helfen, die Erweiterbarkeit für zukünftig geplante Features zu gewähren. Für die Überprüfung konnte Silvan Gehrig gewonnen werden. Seine Rückmeldungen, welche im Anhang \ref{sec:code-review} nachgelesen werden können, wurden wie nachfolgend beschrieben umgesetzt.

Auch konnten wir die in der Studienarbeit beschriebenen Vorschläge erfolgreich in der App umsetzen.
 
\paragraph*{Refactoring Backend}~\\
Dieser Abschnitt ist so aufgebaut, dass anhand von zwei Beispielen aufgezeigt werden soll, wie der Code im Backend vor und nach dem Refactoring ausgesehen hat. Auch wird am Ende des Abschnitts kurz erwähnt, was dies für Verbesserungen zur Folge hatte.

Das nachfolgende Listing \ref{participantErstellenVorRef} zeigt einen Ausschnitt aus der \texttt{Participant\-Controller\-.java} Klasse wie es vor dem Refactoring war. Das Listing \ref{participantErstellenNachRef} zeigt dieselbe Methode nach dem Refactoring.

\lstset{language=JAVA, showstringspaces=false, frame=single, captionpos=b, label=createParticipant, breaklines=true, numbers=left}
\begin{lstlisting}[caption={Participant erstellen vor Refactoring}, label=participantErstellenVorRef]
public Result createParticipant(){

    JsonNode body = request().body().asJson();

    if (body == null ) {
        return forbidden(Json.toJson(new ErrorMessage("Error", "json body is null")));
    } else if(  body.hasNonNull("username") &&
            	body.hasNonNull("password") &&
            	body.hasNonNull("firstname") &&
            	body.hasNonNull("lastname")) {

    Participant participant = new Participant(body.get("username").asText(), body.get("password").asText(), body.get("firstname").asText(), body.get("lastname").asText());

    participantCollection.insertOne(participant, new SingleResultCallback<Void>() {
        @Override
        public void onResult(Void result, Throwable t) {
            Logger.info("Inserted Participant!");
        }
    });

    return ok(Json.toJson(new SuccessMessage("Success", "Participant successfully inserted")));
    }

    return forbidden(Json.toJson(new ErrorMessage("Error", "json body not as expected")));
}
\end{lstlisting}

\begin{lstlisting}[caption={Participant erstellen nach Refactoring}, label=participantErstellenNachRef]
@BodyParser.Of(ParticipantDTOBodyParser.class)
public Result createParticipant() {
    ParticipantDTO participantDTO = request().body().as(ParticipantDTO.class);
    Participant participant = modelsMapper.toParticipant(participantDTO);

    try {
        if (service.insertParticipant(participant)){
            return ok(Json.toJson(new SuccessMessage("Success", "Participant successfully inserted")));
        } else {
            Logger.info("Username already exists");
            return badRequest(Json.toJson(new ErrorMessage("Error", "Username already exists")));
        }
    } catch (ExecutionException | InterruptedException e) {
        return internalServerError(Json.toJson(new ErrorMessage("Error", e.getMessage())));
    }
}
\end{lstlisting}

Das Erstellen des Participants wurde komplett in den \texttt{Participant\-DTO\-Body\-Parser\-.class} ausgelagert. Dieser erstellt nun aus dem angelieferten JSON ein Data-Transfer-Object \cite{DTO}, welches in einem nächsten Schritt zu einem Business Object aufgelöst wird (Zeile 5).

Jegliche Überprüfungen aus Listing \ref{participantErstellenVorRef} (Zeilen 5 und 7-10) konnten so zentral in den BodyParser ausgelagert werden.

Bei den nachfolgenden Listings \ref{putBrainsheetVorRef} und \ref{putBrainsheetNachRef}, welche einen Ausschnitt aus dem \texttt{Brain\-storming\-Finding\-Controller\-.java} zeigen, ist der Unterschied noch stärker zu sehen. So konnte zum Beispiel die \texttt{putBraisheet} Methode von zirka 40 Zeilen auf 15 Zeilen gekürzt werden. 

\begin{lstlisting}[caption={PutBrainsheet vor Refactoring}, label=putBrainsheetVorRef]
public Result putBrainsheet(String findingIdentifier) throws ExecutionException, InterruptedException {

JsonNode body = request().body().asJson();
JsonNode brainwaves = body.findPath("brainwaves");
JsonNode nrOfSheet = body.findPath("nrOfSheet");

if (body == null ) {
    return forbidden(Json.toJson(new ErrorMessage("Error", "json body is null")));
} else if(  !brainwaves.isNull() &&
            !nrOfSheet.isNull()){

	BrainstormingFinding finding = getBrainstormingFinding(findingIdentifier);

    if (finding == null){
        return internalServerError(Json.toJson(new ErrorMessage("Error", "No brainstormingFinding found")));
    }

    Brainsheet oldBrainsheet = finding.getBrainsheets().get(nrOfSheet.asInt());
    Brainsheet newBrainsheet = createBrainsheet(body);


    findingCollection.updateOne(eq("identifier", findingIdentifier),pullByFilter(Filters.eq("brainsheets", oldBrainsheet)), new SingleResultCallback<UpdateResult>() {
        @Override
        public void onResult(final UpdateResult result, final Throwable t) {
            Logger.info(result.getModifiedCount() + " Brainsheet successfully deleted");
        }
    });

    findingCollection.updateOne(eq("identifier", findingIdentifier),combine(pushEach("brainsheets", Arrays.asList(newBrainsheet), new PushOptions().position(newBrainsheet.getNrOfSheet())), inc("deliveredBrainsheetsInCurrentRound", 1)), new SingleResultCallback<UpdateResult>() {
        @Override
        public void onResult(final UpdateResult result, final Throwable t) {
            Logger.info(result.getModifiedCount() + " Brainsheet successfully inserted");
        }
    });

    return ok(Json.toJson(new SuccessMessage("Success", "Brainsheet successfully updated")));
}

return forbidden(Json.toJson(new ErrorMessage("Error", "json body not as expected")));
}
\end{lstlisting}


\begin{lstlisting}[caption={PutBrainsheet nach Refactoring}, label=putBrainsheetNachRef]
@BodyParser.Of(BrainsheetDTOBodyParser.class)
public Result putBrainsheet(String findingIdentifier){
    BrainsheetDTO brainsheetDTO = request().body().as(BrainsheetDTO.class);
    Brainsheet newBrainsheet = modelsMapper.toBrainsheet(brainsheetDTO);

    try {

        if (service.exchangeBrainsheet(findingIdentifier, newBrainsheet)) {
            return ok(Json.toJson(new SuccessMessage("Success", "Brainsheet successfully updated")));
        } else {
            return badRequest(Json.toJson(new ErrorMessage("Error", "No Brainsheet updated")));
        }

    } catch (ExecutionException | InterruptedException e) {
        return internalServerError(Json.toJson(new ErrorMessage("Error", e.getMessage())));
    }
}
\end{lstlisting}

Die gesamte Logik für den Austausch der Brainsheets wurde in die \texttt{FindingService\-.java} Klasse ausgelagert.

\begin{lstlisting}[caption={Exchange Brainsheet im Business Layer}, label=exchangeBrainsheetBusinessLayer]
public boolean exchangeBrainsheet(String findingIdentifier, Brainsheet newBrainsheet) {
BrainstormingFinding finding = service.getFinding(findingIdentifier).get();

if (finding == null){
    return false;
} else {
    if (newBrainsheet.getNrOfSheet() < finding.getBrainsheets().size()) {
        Brainsheet oldBrainsheet = finding.getBrainsheets().get(newBrainsheet.getNrOfSheet());
        service.exchangeBrainsheet(finding, oldBrainsheet, newBrainsheet);
        return true;
    }
    return false;
}

}
\end{lstlisting}

Auch wurde die Implementation für das Einfügen in die Datenbank in eine separate Klasse (\texttt{MongoDBFindingservice.java}) ausgelagert, um eine bessere Übersicht und ein besseres Layering zu gewähren.

\begin{lstlisting}[caption={Exchange Brainsheet im Data Access Layer}, label=exchangeBrainsheetDAL]
public void exchangeBrainsheet(BrainstormingFinding finding, Brainsheet oldBrainsheet, Brainsheet newBrainsheet){
    
    //delete old Brainsheet
    findingCollection.updateOne(eq("identifier", finding.getIdentifier()),pullByFilter(Filters.eq("brainsheets", oldBrainsheet)), (result, t) -> Logger.info(result.getModifiedCount() + " old Brainsheet successfully deleted"));
    
    //insert new Brainsheet at the same place
    findingCollection.updateOne(eq("identifier", finding.getIdentifier()),combine(pushEach("brainsheets", Arrays.asList(newBrainsheet), new PushOptions().position(newBrainsheet.getNrOfSheet())), inc("deliveredBrainsheetsInCurrentRound", 1)), (result, t) -> Logger.info(result.getModifiedCount() + " new Brainsheet successfully inserted"));
}
\end{lstlisting}
Diese zwei Beispiele stellen noch lange nicht alle überarbeiteten Codestellen dar, geben aber eine gute Übersicht, wie der gesamte Code vereinfacht werden konnte.

\paragraph*{Fazit}
Gerade das Beispiel vom Austausch der Brainsheets (Listing \ref{putBrainsheetVorRef}) verdeutlicht, dass das Layering stark verbessert wurde. Vieles was vorher in einer Klasse war, konnte in verschiedene Klassen ausgelagert werden. Damit konnte die Übersicht und Komplexität deutlich verbessert werden.

Auch wurden im gesamten Backend Data-Transfer-Obekte (DTO) \cite{DTO} eingefügt. Dies ermöglicht eine stärkere Trennung von relevanten Informationen für Business-Objekte (BO) und relevanten Informationen für Data-Transfer-Obekte. Dies war vor allem im späteren Verlauf der Arbeit von grossem Wert.

Mit Hilfe der BodyParser-Klassen konnte die komplette Überprüfung und Deserialisierung der angelieferten JSON-Daten zentral geregelt werden. Somit ist immer sichergestellt, dass das DTO korrekt (alle erwarteten Informationen sind vorhanden und das Format stimmt) erstellt wird. 

\paragraph*{Refactoring Frontend}~\\

\subsubsection{Implementation Skizzen Feature}
Nachdem die Refactoring-Phase beendet war, begannen wir die Skizzen-Funktion in unsere Applikation zu integrieren. Diese Funktion sollte es dem Endnutzer ermöglichen, nebst den \texttt{NoteIdeas} auch \texttt{SketchIdeas} (siehe Abbildung \ref{fig:domainmodell-methode635}) zu erfassen. Mit den \texttt{SketchIdeas} wird dem Endnutzer die Möglichkeit gegeben, Skizzen als Teil der Brainstorming-Runden zu zeichnen.

Da es sich bei den einzelnen Skizzen um Bilder handelt, musste zunächst eine Möglichkeit geschaffen werden, diese Bilder in der Datenbank abzulegen. Dafür wurden in der Elaboration-Phase Analysen (siehe Kapitel \ref{seq:save_file_in_db}) durchgeführt und ein Prototyp erstellt, um die technische Machbarkeit zu verifizieren.

\paragraph*{Implementation Backend}~\\
In den nachfolgenden Listing wird aufgezeigt, wie die Sketch-Funktion im Backend umgesetzt wurde.
Der genaue Ablauf kann der Abbildung \ref{fig:Seq-Draw-Sketch} entnommen werden.

\begin{lstlisting}[caption={Upload File im File Controller}, label=uploadFileController]
@BodyParser.Of(MultipartFormDataBodyParser.class)
public Result uploadFile(){
    try {

    final Http.MultipartFormData<File> formData = request().body().asMultipartFormData();
    final Http.MultipartFormData.FilePart<File> filePart = formData.getFile("name");
    final File file = filePart.getFile();

    final byte[] fileData = Files.readAllBytes(file.toPath());
    final String fileName = file.getName();

    String fileId = service.uploadFileAsStream(fileData, fileName);
    return ok(Json.toJson(new SuccessMessage("Success", fileId)));

    } catch (IOException | InterruptedException | ExecutionException  e) {
        return internalServerError(Json.toJson(new ErrorMessage("Error", e.getMessage())));
    }
}
\end{lstlisting}

Ähnlich wie bei den JSON-Daten, wird auch hier zunächst das Bild mittels eines BodyParsers (Zeile 1) in ein File umgewandelt (Zeilen 5-7), welches dann als Byte-Array dem File-Service (Zeile 12) übergeben werden kann. 

\begin{lstlisting}[caption={Upload File im File Service}, label=uploadFileService]
public String uploadFileAsStream(byte[] stream, String fileName) ... {
    return service.uploadFileAsStream(stream,fileName);
}
\end{lstlisting}

Der File-Service nimmt diesen Stream entgegen und gibt diesen unverändert dem Datenbank-Service weiter.

\begin{lstlisting}[caption={Upload File im DB Service}, label=uploadFileDBService]
@Override
public String uploadFileAsStream(byte[] stream, String fileName) ... {
    ByteBuffer data = ByteBuffer.wrap(stream);
    CompletableFuture<String> future = new CompletableFuture<>();

    final GridFSUploadStream uploadStream = gridFSBucket.openUploadStream(fileName);
    uploadStream.write(data, (result, t) -> {
        Logger.info("File successfully inserted; ID: " + uploadStream.getObjectId().toHexString());
        future.complete(uploadStream.getObjectId().toHexString());

        uploadStream.close((result1, t1) -> {
            // stream close
        });
    });

    return future.get();
}
\end{lstlisting}

Der Datenbank-Service speichert anschliessend den Stream in die Datenbank (Zeile 7) und liefert bei erfolgreicher Speicherung die ObjektID zurück (Zeile 9). Am Ende wird noch der uploadStream zur Datenbank geschlossen (Zeile 11).

Die Smartphone-Applikation speichert nun die ObjektID als Teil einer \texttt{SketchIdea} und sendet das \texttt{Brainsheet} wie gewohnt nach Ablauf der Zeit an das Backend.

Das Herunterladen der gespeicherten Bilder funktioniert auf ganz ähnliche Weise. Da die ObjektID in der \texttt{SketchIdea} abgelegt ist, kann das eigentliche Bild problemlos wiedergefunden werden.

\begin{lstlisting}[caption={Download File im DB Service}, label=uploadFileDBService]
@Override
public byte[] downloadFileAsStream(String id) ...{
    ObjectId fileId = new ObjectId(id);
    final ByteBuffer dstByteBuffer = ByteBuffer.allocate(1024 * 1024);
    final GridFSDownloadStream downloadStream = gridFSBucket.openDownloadStream(fileId);
    CompletableFuture<byte[]> future = new CompletableFuture<>();

    downloadStream.read(dstByteBuffer, (result, t) -> {
        dstByteBuffer.flip();
        byte[] bytes = new byte[result];
        dstByteBuffer.get(bytes);
        Logger.info("Found file to download; Size: " + result);
        future.complete(bytes);

        downloadStream.close((result1, t1) -> {
            // stream closed
        });
    });

    return future.get();
}
\end{lstlisting}

Hierbei wird ein downloadStream geöffnet, welcher die Daten mittels ID aus der Datenbank liest und in ein Byte-Array schreibt (Zeile 8). Auch hier wird am Ende der Stream wieder geschlossen (Zeile 15).

Der zurückgelieferte Byte-Array wird anschliessend unverändert an den Client zurückgeschickt. Allfällige Fehler während der Ausführung sowohl beim Hochladen wie auch beim Herunterladen werden im File-Controller abgefangen und als Fehler an den Client geschickt.

\paragraph*{Implementation Frontend}~\\

\subsubsection{Implementation Pattern Feature}
\paragraph*{Implementation Backend}

\paragraph*{Implementation Frontend}

\subsubsection{Implementation Export Feature}
\paragraph*{Implementation Backend}

\paragraph*{Implementation Frontend}



\subsubsection{Verwendete Bibliotheken im Backend}
Tabelle \ref{tab:verwendete-libraries-play} zeigt die Bibliotheken auf, die im Backend verwendet werden. 
\begin{table}[h]
	\centering
	\begin{tabular}{| l | l | c | l |}
		\hline
		\textbf{Bibliothek} & \textbf{Version} & \textbf{Repository} & \textbf{Lizenz}\\
		\hline
		swagger-play2 & 1.6.0 & \href{https://mvnrepository.com/artifact/io.swagger/swagger-play2_2.12/1.6.0}{mvnrepository.com} & Apache 2.0 \\
		java-jwt & 3.2.0 & \href{https://mvnrepository.com/artifact/com.auth0/java-jwt/3.2.0}{mvnrepository.com} & MIT \\
		mongodb-driver-async & 3.8.0 & \href{https://mvnrepository.com/artifact/org.mongodb/mongodb-driver-async/3.8.0}{mvnrepository.com} & MIT \\
		modelmapper & 2.3.2 & \href{https://mvnrepository.com/artifact/org.modelmapper/modelmapper/2.3.2}{mvnrepository.com} & Apache 2.0 \\
		markdowngenerator & 1.3.1.1 & \href{https://mvnrepository.com/artifact/net.steppschuh.markdowngenerator/markdowngenerator}{mvnrepository.com} & Apache 2.0 \\
		\hline
	\end{tabular}
	\caption[Story-Points]{Verwendete Bibliotheken Backend}
	\label{tab:verwendete-libraries-play}
\end{table}

\subsubsection{Verwendete Bibliotheken im Frontend}
In Tabelle \ref{tab:verwendete-libraries-frontend} sind die verwendeten Libraries und Frameworks des Frontends aufgelistet.
\begin{table}[!h]
	\centering
	\begin{tabular}{| l | l | c | l |}
		\hline
		\textbf{Bibliothek} & \textbf{Version} & \textbf{Repository} & \textbf{Lizenz}\\
		\hline
		CarouselView.FormsPlugin & 5.2.0 & \href{https://github.com/alexrainman/CarouselView}{github.com} & MIT \\
		Microsoft.AppCenter & 1.10.0 & \href{https://visualstudio.microsoft.com/app-center/}{AppCenter} & MIT \\
		NUnit & 3.11.0 & \href{http://nunit.org}{nunit.org} & MIT\\
		NUnit3TestAdapter & 3.11.2 & \href{https://github.com/nunit/docs/wiki/Visual-Studio-Test-Adapter}{github.com} & MIT\\
		Prism.Forms & 7.0.0.396 & \href{https://github.com/PrismLibrary/Prism}{github.com} & MIT \\
		Xamarin.Forms & 3.3.0.912540 & \href{https://docs.microsoft.com/en-us/xamarin/xamarin-forms/}{Microsoft Docs} & MIT \\
		XamlStyler.Console & 3.0.0 & 
		\href{https://github.com/Xavalon/XamlStyler}{github.com} & Apache 2.0\\
		ZXing.Net.Mobile & 2.4.1 & \href{http://github.com/Redth/ZXing.Net.Mobile}{github.com} & Apache 2.0\\
		ZXing.Net.Mobile.Forms & 2.4.1 &
		\href{http://github.com/Redth/ZXing.Net.Mobile}{github.com} & Apache 2.0\\
		\hline
	\end{tabular}
	\caption{Direkt verwendete Bibliotheken Frontend}
	\label{tab:verwendete-libraries-frontend}
\end{table}


\subsubsection{Vergleich Soll/Ist}
%TODO
\newpage

\subsection{Schlussfolgerungen}
Im Kapitel der Schlussfolgerungen wollen wir nochmals auf unser Projekt und dessen Verlauf schauen und unsere Ergebnisse kritisch bewerten. Dabei wollen wir aufzeigen, was wir in dieser Zeit erreicht haben, aber auch an welchen Stellen es noch Verbesserungspotenzial gibt. Ausserdem soll im Kapitel \ref{subsub:Ausblick} aufgezeigt werden, was das weitere Vorgehen für dieses Projekt sein könnte.

\subsubsection{Ergebnisbewertung}

\subsubsection{Bekannte Probleme}

\subsubsection{Ausblick}
\label{subsub:Ausblick}
Da die Ausbaumöglichkeiten dieses Projektes sehr vielfältig sind, empfinden wir es als sehr lohnenswert diese Applikation zu erweitern.

Dazu sehen wir drei mögliche Arten oder Wege, wie die bestehende Applikation verbessert bzw. erweitert werden könnte. Diese wären weitere Ideen-Typen zu integrieren, die erarbeiteten Ideen stärker miteinander zu referenzieren oder die Methode 635 mit der World-Cafe Methode zu kombinieren.

\paragraph*{Weitere Ideen-Typen}~\\
Die einfachste und naheliegenste Erweiterung könnte darin bestehen, die Applikation um weitere Ideen-Typen zu erweitern. Die gesamte Applikation ist so programmiert, dass dies mit möglichst wenig Aufwand machbar ist (siehe Anhang \ref{sec:Ideen_Erweiterung}). 

Dabei könnten wir uns vorstellen, dass sich Ideen-Typen wie QualityIdea (angelehnt an nicht-funktionale Anforderungen), RequirementIdea (angelehnt an funktionale Anforderungen)  oder CodeIdea relativ gut umsetzen liessen.

\paragraph*{Ideen stärker referenzieren}~\\
Bei verschiedensten Durchführungen der Methode 635 auf dem Papier, kam immer wieder die Frage auf, wie ich den anderen Teilnehmern mitteilen kann, auf welche Idee ich gerade Bezug nehme. Dies ist in der vorliegenden Applikation nicht möglich bzw. nur indem man sich darauf einigt, dass die Reihenfolge der Ideen ausschlaggebend ist.

Daher wäre eine weitere Ausbaumöglichkeit, dem Endnutzer die Möglichkeit zu geben seine Ideen stärker mit den vorliegenden Ideen zu 'verknüpfen' oder zu referenzieren. Dies würde auch das Problem der oben geschilderten Unsicherheit lösen.

\paragraph*{Methode World-Cafe}~\\
Als weitere mögliche Erweiterung ist es vorstellbar, die Methode 'World-Cafe' \cite{world-cafe} in die bestehende Applikation zu integrieren. Im Gegensatz zur Methode 635 sind hier nicht einzelne Teilnehmer, welche ihre Ideen zu einem bestimmten Problem schildern, sondern ganze Gruppen.

Das Konzept von World-Cafe sieht daher vor, dass man sich in einer Gruppe zu einer bestimmten Fragestellung austauscht. Ist eine vordefinierte Zeitspanne abgelaufen, wechseln die Teilnehmer zu einer anderen Gruppe, um dort wieder eine andere oder auch die gleiche Fragestellung zu diskutieren. 

Um die World-Cafe Methode in die bestehende Applikation zu integrieren, ist es an dieser Stelle aber ratsam von der ursprünglichen Version abzuweichen und die Gruppen immer gleich zu belassen. 

Das hätte zur Folge, dass mehrere Teams (statt mehrere Participants) gemeinsam ein BrainstormingFinding erarbeiten. Der Umstand dass mehrere Teams ein BrainstormingFinding lösen führt dazu, dass innerhalb eines Teams schneller verschiedenste Ideen entstehen, als wenn ein Participant alleine Ideen/Lösungsansätze entwickelt. 

Dies würde der gesamten Applikation nochmals einen Mehrwert bieten. 

\newpage

