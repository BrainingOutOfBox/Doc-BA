\section{Persönliche Berichte}
Um über den Projektverlauf zu reflektieren, schreiben in diesem Kapitel beide Teilnehmer ihre eigenen Erfahrungen auf und beschreiben Besonderheiten, die ihnen während dem Projekt begegnet sind. 

\subsection{Oliver Dias}

\subsection{Elias Brunner}
Die Bachelorarbeit ist aus meiner Sicht sehr gut verlaufen. Das dem so war, hat meiner Meinung nach mehrere Gründe. Da Oliver und ich die vorangegangene Studienarbeit zusammen durchgeführt hatten, kannten wir uns schon ziemlich gut und wussten wie die Arbeitshaltung des jeweils Anderen ist. Wie schon in der Studienarbeit empfand ich die Zusammenarbeit mit ihm daher als sehr angenehm.

Auch konnten wir unseren Betreuer, Prof. Dr. Zimmermann, für eine Weiterführung der bestehenden Arbeit begeistern. Dies kam dem Projekterfolg natürlich insofern zu Gute als das wir auch ihn schon etwas kannten und wussten, was ihm in solchen Projekten wichtig ist und auf was wir besonders Wert legen mussten. Die Kommunikation sowohl zwischen unserem Betreuer wie auch mit Oliver empfand ich als sehr gut.

Da wir auch hier, wie schon bei der Studienarbeit, die Arbeiten nach unseren Stärken aufteilten bzw. die Aufteilung der Arbeiten aus Gründen der Einfachheit beibehielten, kam ich auch in dieser Arbeit nur begrenzt in Kontakt mit Xamarin. Kritisch betrachtet würde ich daher sagen, dass ich mir nur begrenzt Erfahrung in diesem Gebiet aneignen konnten. 

Als es aber darum ging unsere Cross-Platform Applikation für einen User Test anderen Nutzern (vor allem iOS Nutzer) zur Verfügung zu stellen, konnte ich vieles über das Deployment und das Signieren einer Applikation für iOS Geräte lernen. Ich kann nun mit Bestimmtheit sagen, dass es sich einfacher anhört als es tatsächlich ist, vor allem wenn man es das erste Mal macht.

Wenn ich nun auf die vergangenen Wochen zurückschaue, bin ich schon etwas stolz darauf, dass wir fast alle Features, welche wir uns vorgenommen hatten auch umsetzen konnten. Am besten finde ich es, dass unsere App nun auch stabil und flüssig läuft. So können auch andere Leute unsere Applikation produktiv nutzen.
 