\section{Persönliche Berichte}
Um über den Projektverlauf zu reflektieren, schreiben in diesem Kapitel beide Teilnehmer ihre eigenen Erfahrungen auf und beschreiben Besonderheiten, die ihnen während dem Projekt begegnet sind. 

\subsection{Oliver Dias}

Nach einer intensiven und lehrreichen Zeit kann ich positiv auf diese Bachelorarbeit zurückschauen. Besonders viel Wissen konnte ich mir während der Entwicklungsphase aneignen, da das Refactoring und Implementieren der neuen Features viele herausfordernde Situationen hervorbrachte, die aber meiner Meinung nach alle gut gelöst werden konnten.  

Es zahlte sich besonders aus, während Zeiten, in welchen die Applikation nicht tat was sie sollte, mehr aufzuwenden und am Ball zu bleiben. Dass wir dies taten kam uns besonders gegen Ende der Arbeit zugute, denn da gab es keine hektischen Situationen mehr und wir konnten die App wie geplant abschliessen.

Wie schon während der Studienarbeit hat sich die Zusammenarbeit mit Elias und Prof. Dr. Zimmermann als sehr angenehm herausgestellt. Wir konnten uns immer auf Problemlösungen einigen und ich konnte mich jederzeit auf die Unterstützung beider Parteien verlassen. Dies und die gute Kommunikation im Team sowie mit dem Betreuer vereinfachte den Projektverlauf wesentlich und trug viel zum Projekterfolg bei. 

Durch das noch gründlichere Auseinandersetzen mit Xamarin Forms bin ich zum Schluss gekommen, dass das Framework für Cross-Plattform-Apps zufriedenstellend ist und einiges an Aufwand eingespart werden kann, wenn man es mit dem Aufwand von zwei plattform-spezifischen Implementation vergleicht. Die Dokumentation dazu ist ausführlich und hilft in den meisten Fällen weiter. 

Ich denke, dass wir für diese Arbeit mit unserem Wissensstand die sinnvollsten Entscheide getroffen haben und ich würde in einem zukünftigen Projekt ähnlich vorgehen. Diese Erkenntnis und das Endprodukt macht mich durchaus stolz.

\subsection{Elias Brunner}
Die Bachelorarbeit ist aus meiner Sicht sehr gut verlaufen. Dass dem so war, hat meiner Meinung nach mehrere Gründe. Da Oliver und ich die vorangegangene Studienarbeit zusammen durchgeführt hatten, kannten wir uns schon ziemlich gut und wussten wie die Arbeitshaltung des jeweils Anderen ist. Wie schon in der Studienarbeit empfand ich die Zusammenarbeit mit ihm daher als sehr angenehm.

Auch konnten wir unseren Betreuer, Prof. Dr. Zimmermann, für eine Weiterführung der bestehenden Arbeit begeistern. Dies kam dem Projekterfolg natürlich insofern zugute als das wir auch ihn schon etwas kannten und wussten, was ihm in solchen Projekten wichtig ist und auf was wir besonders Wert legen mussten. Die Kommunikation sowohl zwischen unserem Betreuer wie auch mit Oliver empfand ich als sehr gut.

Da wir auch hier, wie schon bei der Studienarbeit, die Arbeiten nach unseren Stärken aufteilten bzw. die Aufteilung der Arbeiten aus Gründen der Einfachheit beibehielten, kam ich auch in dieser Arbeit nur begrenzt in Kontakt mit Xamarin. Kritisch betrachtet würde ich daher sagen, dass ich mir nur begrenzt Erfahrung in diesem Gebiet aneignen konnten. 

Als es aber darum ging unsere Cross-Platform Applikation für einen User Test anderen Nutzern (vor allem iOS Nutzer) zur Verfügung zu stellen, konnte ich vieles über das Deployment und das Signieren einer Applikation für iOS Geräte lernen. Ich kann nun mit Bestimmtheit sagen, dass es sich einfacher anhört als es tatsächlich ist, vor allem wenn man es das erste Mal macht.

Wenn ich nun auf die vergangenen Wochen zurückschaue, bin ich schon etwas stolz darauf, dass wir fast alle Features, welche wir uns vorgenommen hatten auch umsetzen konnten. Am besten finde ich es, dass unsere App nun auch stabil und flüssig läuft. So können auch andere Leute unsere Applikation produktiv nutzen.
 