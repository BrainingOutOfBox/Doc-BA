\section{Installationsanleitung}

\subsection{Zweck dieses Dokuments}
Die Installationsanleitung bietet interessierten Personen eine Schritt-für-Schritt Anleitung, wie das gesamte System installiert wird. Da sich die Installationsanleitung nicht wesentlich von jener aus unserer Studienarbeit \cite{methode635-sa} unterscheidet, konnten viele Teile übernommen werden. Dennoch wurde diese Installationsanleitung etwas angepasst und präzisiert. 
 
\subsection{Anforderungen}
Für die Installation der Cross-Plattform Applikation auf dem eigenen Smartphone wird Folgendes vorausgesetzt:

\begin{itemize}
  \item Visual Studio IDE Community Edition (7.6.11 (build 9))
  \item Android + Xamarin.Forms
  \item iOS + Xamarin.Forms
  \item .NET Core + ASP.Net Core
\end{itemize}
Die Visual Studio IDE kann unter \href{https://visualstudio.microsoft.com/de/vs/}{www.visualstudio.microsoft.com} heruntergeladen werden. 

Wem es nur darum geht, die Xamarin Applikation auszuprobieren, kann bei Kapitel \ref{subsub:installation} weiterlesen. Wer allerdings zusätzlich noch das PlayFramework laufen lassen möchte, benötigt zusätzlich noch Folgendes:

\begin{itemize}
  \item Java SE 1.8 oder höher
  \item Ein Build-Tool wie SBT oder Gradle
  \item Docker Community Edition
\end{itemize}
Weitere Informationen zur Installation von Java oder den Build-Tools sind unter \href{https://www.playframework.com/documentation/2.6.x/Requirements}{www.play\-frame\-work.com} zu finden.


\subsection{Installation}
Bevor die eigentliche Installation jedoch beginnen kann, muss man noch über den Quellcode verfügen. Dieser kann von unseren \href{https://github.com/BrainingOutOfBox}{Github Repositories} heruntergeladen werden.

\subsubsection*{Installation der Cross-Plattform Applikation}
\label{subsub:installation}
Für die Installation der Xamarin Applikation auf einem Android Smartphone muss folgendermassen vorgegangen werden.

\begin{enumerate}
  \item Öffne das Projekt mit Visual Studio.
  \item Wähle das \grqq Method635.App.Forms.Android \grqq als Startprojekt.
  \item Schliesse dein Smartphone mittels USB-Kabel an deine Entwicklungshardware an.
  \item Mittels F5 oder dem Play Button kann die Applikation auf dem Smartphone gestartet werden.
\end{enumerate}
Für die Installation der Xamarin Applikation auf einem iOS Smartphone muss etwas anders vorgegangen werden. Dabei sind allerdings ein Entwickler Account von Apple sowie ein Apple Mac notwendig.

\begin{enumerate}
  \item Öffne zuerst Xcode und erstelle ein leeres Projekt. Trage dabei 'ch.brainingoutofbox\-.method635' als Bundle Identifier ein.
  \item Schliesse dein Smartphone mittels USB-Kabel an deine Entwicklungshardware an.
  \item Signiere das leere Projekt und stelle es auf deinem iOS Gerät bereit. Dieser Vorgang erstellt auch ein sogenanntes 'Provisioning Profile'. 
  \item Öffne nun das Method635-Projekt mit Visual Studio.
  \item Wähle das \grqq Method635.App.Forms.iOS \grqq als Startprojekt.
  \item Navigiere zur Datei Info.plist und öffne diese.
  \item Wähle nun 'Manuelle Bereitstellung'. Unter den Optionen für die Signierung wählst du nun deinen Apple Entwickler Account aus und das zuvor erstellte Provisioning Profile.
  \item Mittels F5 oder dem Play Button kann die Applikation auf dem Smartphone gestartet werden. 
  \end{enumerate}

\textbf{Bemerkung:} Sollte dabei der Fehler \grqq resource fork, Finder information, or similar detritus not allowed\grqq{} auftreten, kann dieser wie auf \href{https://stackoverflow.com/questions/39652867/code-sign-error-in-macos-high-sierra-xcode-resource-fork-finder-information}{www.stackoverflow.com} beschrieben, vermieden werden. Danach muss das Projekt aber zuerst bereinigt werden.

\subsubsection*{Installation des PlayFrameworks}
Für die Installation des PlayFrameworks muss zuerst eine Datenbank mit den entsprechenden Collections bereitgestellt werden. Dazu nutzen wir eine MongoDB-Instanz, welche in einem Docker-Container läuft.

\begin{enumerate}
  \item docker run -name methode635 -p 40002:27017 -d mongo:latest
  \item docker exec -it methode635 bash
  \item mongo --username root --password --authenticationDatabase admin --host localhost --port 40002
  \item use Methode635
  \item db.createCollection("BrainstormingFinding")
  \item db.createCollection("BrainstormingTeam")
  \item db.createCollection("Participant")
  \item db.createCollection("Pattern")
\end{enumerate}

Weitere Informationen zu MongoDB als Docker-Container können unter \href{https://hub.docker.com/_/mongo/}{hub.docker.com} nachgelesen werden. Wer mehr über die createCollection Befehle erfahren möchte, findet unter \href{https://docs.mongodb.com/manual/reference/method/db.createCollection/index.html}{docs.mongodb.com} mehr dazu.

Um das PlayFramework zu builden und laufen zu lassen genügt es, im API Projekt-Ordner folgenden Befehl auszuführen.

\begin{enumerate}
  \item sbt run "40000"
\end{enumerate}

Weitere Hilfestellungen im Umgang mit dem Simple Build Tool (SBT) und dem PlayFramework finden sich auf \href{https://www.playframework.com/documentation/2.6.x/PlayConsole}{playframework.com}. Um die Kommunikation zwischen iOS Geräten und dem Backend aufzubauen, ist es zwingend notwendig ein SSL Zertifikat zu verwenden. Nähere Angaben dazu können auf \href{https://www.playframework.com/documentation/2.6.x/ConfiguringHttps#SSL-Certificates}{www.playframework.com} nachgelesen werden.


\textbf{Bemerkung:} Allenfalls muss für die Interaktion zwischen API und der Datenbank der Username, das Passwort sowie der Datenbankname geändert werden. Dies kann in der 'application.conf' Datei angepasst werden.

\lstset{language=JAVA, showstringspaces=false, frame=single, captionpos=b, label=createParticipant, breaklines=true, numbers=left}
\begin{lstlisting}[caption={Verbindung zur MongoDB}, label=DBVerbindung]
db {
# You can declare as many datasources as you want.
  mongo.url = "localhost"
  mongo.port = "40002"
  mongo.database = "someDatabase"
  mongo.username = "username"
  mongo.password = "password"

}

\end{lstlisting}

Auch muss im Quellcode der Cross-Plattform Applikation der URL auf die IP deines PC gewechselt werden. Dies kann in der 'backend-config.json' Datei geändert werden.

\begin{lstlisting}[caption={Verbindung zum API}, label=APIVerbindung]
"server": {
    "hostname": "IPOfYourPC",
    "port": 40000
}
\end{lstlisting}

