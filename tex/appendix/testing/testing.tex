\section{Testprotokoll}
\subsection{Zweck dieses Dokuments}
Um sicher zu stellen, dass unsere Cross-Plattform Applikation in ihrer Kernfunktionalität richtig funktioniert, wurde ein systematischer Test anhand des nachfolgenden Testprotokolls durchgeführt.

\subsection{Verweise}
Die Testfälle leiten sich von den Fully-Dressed Use Cases aus Kapitel \ref{par:fully-dressed-uc} ab.

\subsection{Testaufbau}

\subsection{Testfälle}

\renewcommand{\arraystretch}{1.35}
\begin{center}
	\begin{longtable}{| p{1cm} | p{4cm} | p{5cm} | p{3cm} |}
		\hline
		\multicolumn{4}{|c|}{\textbf{Use Case 7: View Brainstorming Finding}}\\
		\hline\hline
		Test Nr & Beschreibung & Erwartetes Resultat & Beo\-bach\-te\-tes Verhalten (Bestanden?) \\
		\hline
		1 & Als Participant möchte ich als Letzter die letzte Runde abschliessen. & Das System speichert meine Notizen auf dem Server und zeigt mir eine Meldung an, dass das Finding einsehbar ist. & TODO. \\
		\hline
		2 & Als Participant möchte ich die Resultate der Gruppe ansehen. & Das System zeigt mir eine Übersicht mit allen Notizen der Teilnehmer. & TODO.\\
		\hline
		3 & Als Participant möchte ich nicht als Letzter die letzte Runde abschliessen. & Das System speichert meine Notizen auf dem Server und zeigt mir die verbleibende Zeit an. & TODO.\\
		\hline
		4 & Nach Ablauf der Zeit meldet mir das System, dass das Finding einsehbar ist. & Das System zeigt mir eine Übersicht mit allen Notizen der Teilnehmer. & TODO.\\
		\hline
		5 & Als Participant möchte ich nicht direkt zu den Resultaten der Gruppe navigieren sondern zurück zum Homescreen gelangen. & Das System zeigt mir den Homescreen an. Von dort aus gelange ich aber auch zu den Notizen der Teilnehmer. & TODO.  \\
		\hline
	\end{longtable}
\end{center}

\renewcommand{\arraystretch}{1.35}
\begin{center}
	\begin{longtable}{| p{1cm} | p{4cm} | p{5cm} | p{3cm} |}
		\hline
		\multicolumn{4}{|c|}{\textbf{Use Case 8: Create Brainwave}}\\
		\hline\hline
		Test Nr & Beschreibung & Erwartetes Resultat & Beo\-bach\-te\-tes Verhalten (Bestanden?) \\
		\hline
		1 & Als Participant möchte ich Ideen während der laufenden Rundenzeit zum aktuellen Problem erfassen. & Das System zeigt meine Ideen an. & TODO. \\
		\hline
		2 & Als Participant möchte ich meine Ideen frühzeitig abgeben. & Das System persistiert meine Ideen und zeigt mir eine Bestätigung an. & TODO.\\
		\hline
		3 & Als Participant möchte ich meine Ideen nicht frühzeitig abgeben. & Das System zeigt mir eine Meldung an, dass die Zeit der aktuellen Runde abgelaufen ist. Es persistiert die bis zu dem Zeitpunkt erfassten Ideen. & TODO. \\
		\hline
	\end{longtable}
\end{center}


\renewcommand{\arraystretch}{1.35}
\begin{center}
	\begin{longtable}{| p{1cm} | p{4cm} | p{5cm} | p{3cm} |}
		\hline
		\multicolumn{4}{|c|}{\textbf{Use Case 8c: Draw Sketch}}\\
		\hline\hline
		Test Nr & Beschreibung & Erwartetes Resultat & Beo\-bach\-te\-tes Verhalten (Bestanden?) \\
		\hline
		1 & Als Participant möchte ich eine Skizze während der laufenden Rundenezit zum aktuellen Problem erfassen. & Das System speichert meine Skizze auf dem Server und zeigt mir eine Meldung an, dass die Skizze gespeichert wurde. & TODO. \\
		\hline
	\end{longtable}
\end{center}

\renewcommand{\arraystretch}{1.35}
\begin{center}
	\begin{longtable}{| p{1cm} | p{4cm} | p{5cm} | p{3cm} |}
		\hline
		\multicolumn{4}{|c|}{\textbf{Use Case 8e: Insert Pattern}}\\
		\hline\hline
		Test Nr & Beschreibung & Erwartetes Resultat & Beo\-bach\-te\-tes Verhalten (Bestanden?) \\
		\hline
		1 & Als Participant möchte ich ein Pattern während der laufenden Rundenezit zum aktuellen Problem auswählen. & Das System speichert das ausgewählte Pattern und zeigt mir eine Meldung an, dass das Pattern in die Brainwave eingefügt wurde. & TODO. \\
		\hline
	\end{longtable}
\end{center}

\subsection{Testauswertung}


