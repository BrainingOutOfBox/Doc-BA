\section{Anleitung für Ideen-Erweiterungen}
\label{sec:Ideen_Erweiterung}

\subsection{Zweck dieses Dokuments}
Dieses Dokument stellt eine Anleitung für eine allfällige Erweiterung der vorhandenen Ideen-Typen dar. Ziel dieses Dokumentes ist es demnach, Hilfestellungen in solch einem Fall zu bieten. 

\subsection{Szenario}
Um diese Anleitung etwas einfacher und konkreter zu gestallten, gehen wir von einem bestimmten Szenario aus. Das Szenario sieht vor, dass man sich dazu entschieden hat die Applikation um den Ideen-Typ 'QualityIdea' zu erweitern.

Alle nachfolgenden Anpassungen beziehen sich daher auf dieses Szenario. Andere Ideen-Typen sind aber analog zu integrieren.

\subsection{Anpassungen am Backend}
\begin{labeling}{Zweiter Schritt:}
	\item [Erster Schritt:] Der erste Schritt im Backend besteht darin die entsprechenden Business-Objekte und Data-Transfer-Objekte mit all deren Attributen und Methoden anzulegen. Wichtig hierbei ist es, dass die neu erstellten Klassen von \texttt{Idea} bzw. \texttt{IdeaDTO} erben.
	\item [Zweiter Schritt:] Ist dies geschafft, muss dem JSON-Serialisierer noch beigebracht werden, dass ein weiterer Subtyp hinzugefügt wurde. Hierzu muss in der \texttt{IdeaDTO}-Klasse ein weiterer JsonSubTyp hinzugefügt werden.
	
\begin{lstlisting}[caption={weiterer Ideen-Typ hinzufügen}, label=addIdeaType]
@JsonTypeInfo(...)
@JsonSubTypes({
        @JsonSubTypes.Type(value = NoteIdeaDTO.class, name = "noteIdea"),
        ...,
        @JsonSubTypes.Type(value = QualityIdeaDTO.class, name = "qualityIdea"),
        
})
\end{lstlisting}

	\item [Dritter Schritt:]
	
\end{labeling}





 


\subsection{Anpassungen am Frontend}