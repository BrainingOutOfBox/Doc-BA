\section{Anleitung für Ideen-Erweiterungen}
\label{sec:Ideen_Erweiterung}

\subsection{Zweck dieses Dokuments}
Dieses Dokument stellt eine Anleitung für eine allfällige Erweiterung der vorhandenen Ideen-Typen dar. Ziel dieses Dokumentes ist es demzufolge, Hilfestellungen in solch einem Fall zu bieten. 

\subsection{Szenario}
Um diese Anleitung etwas einfacher und konkreter zu gestalten, gehen wir von einem bestimmten Szenario aus. Das Szenario sieht vor, dass man sich dazu entschieden hat die Applikation um den Ideen-Typ 'QualityIdea' zu erweitern. Bei einer 'QualityIdea' geht es dabei um nicht-funktionale Anforderungen, die sich an den ISO 25010 Standard halten (siehe Abbildung \ref{fig:ISO 25010} in Kapitel \ref{subsub:nfr}). So könnte man sich vorstellen, dass ähnlich wie bei den 'PatternIdeas' eine Auswahlliste von nicht-funktionalen Anforderungen bereitgestellt wird. 

Alle nachfolgenden Anpassungen beziehen sich daher auf dieses Szenario. Andere Ideen-Typen sind aber analog zu integrieren.

\subsection{Anpassungen am Backend}
\begin{labeling}{Zweiter Schritt:}
	\item [Erster Schritt:] Der erste Schritt im Backend besteht darin die entsprechenden Business-Objekte und Data-Transfer-Objekte mit all deren Attributen und Methoden anzulegen. Wichtig hierbei ist es, dass die neu erstellten Klassen von \texttt{Idea} bzw. \texttt{IdeaDTO} erben.
	\item [Zweiter Schritt:] Ist dies geschafft, muss dem JSON-Serialisierer noch gekannt gegeben werden, dass ein weiterer Subtyp hinzugefügt wurde. Hierzu muss in der \texttt{IdeaDTO}-Klasse ein weiterer JsonSubTyp hinzugefügt werden (Zeile 5).
	
\begin{lstlisting}[caption={Weiterer Ideen-Typ hinzufügen in der IdeaDTO-Klasse}, label=addIdeaType]
@JsonTypeInfo(...)
@JsonSubTypes({
    @JsonSubTypes.Type(value = NoteIdeaDTO.class, name = "noteIdea"),
    ...,
    @JsonSubTypes.Type(value = QualityIdeaDTO.class, name = "qualityIdea")
        
})
\end{lstlisting}

	\item [Dritter Schritt:] In einem letzten Schritt muss nur noch die \texttt{ModelsMapper}-Klasse etwas angepasst werden (Zeile 4/6-9 bzw. 14/16-19). 

\begin{lstlisting}[caption={Weiterer Ideen-Typ hinzufügen in der ModelsMapper-Klasse}]
/* FROM BO TO DTO */
modelMapper.createTypeMap(Idea.class, IdeaDTO.class)
    ...
    .include(QualityIdea.class, IdeaDTO.class);

modelMapper.typeMap(QualityIdea.class, IdeaDTO.class).setProvider(request -> {
    QualityIdea idea = (QualityIdea)request.getSource();
    return new QualityIdeaDTO(idea.getDescription(), ...);
});

/* FROM DTO TO BO */
modelMapper.createTypeMap(IdeaDTO.class, Idea.class)
    ...
    .include(QualityIdeaDTO.class, Idea.class);

modelMapper.typeMap(QualityIdeaDTO.class, Idea.class).setProvider(request -> {
    QualityIdeaDTO idea = (QualityIdeaDTO)request.getSource();
    return new QualityIdea(idea.getDescription(), ...);
});
\end{lstlisting}
\end{labeling}

Diese drei Schritte reichen aus, damit das Backend auch mit einer QualityIdea interagieren kann. 

\subsection{Anpassungen am Frontend}

Das Hinzufügen von neuen Ideentypen kann im Frontend in drei Schritte ausgeführt werden:
\begin{enumerate}
	\item Erweiterung des Datenmodells
	\item Ansicht für das Einfügen des neuen Ideentyps
	\item Anzeigen der eingefügten Idee in den Sheets
\end{enumerate}
Die nachfolgenden Unterkapitel beschreiben die aufgelisteten Schritte.

\subsubsection{Erweiterung des Datenmodells}

\begin{labeling}{Zweiter Schritt:}
	\item[Erster Schritt:] Der erste Schritt ist analog zum Backend; die benötigte Idee muss als Business-Objekt sowie Daten-Transfer-Objekt erstellt werden. Auch hier ist es notwendig, von der abstrakten Klasse \texttt{Idea} bzw. \texttt{IdeaDto} abzuleiten. 
	\item[Zweiter Schritt:] Ebenfalls muss die neue Idee dem Mapper bekannt gemacht werden. Dazu ist der Methode \texttt{MapIdeas()} in der Klasse \texttt{Brainstorming\-Mapping\-Profile} das Mapping von und zum DTO zu erstellen und mit \texttt{In\-clude<>()} der \texttt{Idea} zu registrieren. Bei Unklarheiten kann die ausführliche \href{https://docs.automapper.org/en/stable/}{Dokumentation von Automapper} weiterhelfen.
	\item[Dritter Schritt:] Damit die JSON-Serialiserung wie gewünscht funktioniert, muss in der Klasse \texttt{IdeaConverter} das JSON-Attribut 'Type' gelesen und geschrieben werden. Dazu muss ein neuer \texttt{const string} mit Wert 'qualityIdea' initialisiert werden und im Switch-Case der \texttt{ReadJson()}-Methode, analog zu den anderen Ideen, das deserialisierte Objekt zurückgegeben werden. Zusätzlich wird in der \texttt{WriteJson()}-Methode ein weiteres \texttt{else if} benötigt, das den entsprechenden String der Konstante ins 'Type'-Attribut schreibt.
\end{labeling}

Mit diesen Schritten ist das Datenmodell erfolgreich erweitert.

\subsubsection{Ansicht fürs Einfügen von neuen Ideentypen}
Hierbei geht es um die Möglichkeit, mit einer neuen Page die \texttt{QualityIdea} einfügen zu können.

\begin{labeling}{Zweiter Schritt:}
	\item[Erster Schritt:]	Es muss eine neue Prism View und ein neues ViewModel mit der gewünschten Ansicht fürs Einfügen einer \texttt{QualityIdea} erstellt werden. Wichtig ist, dass das ViewModel den \texttt{Brainstorming\-Service} kennt. Dieser wird benötigt, um die Idee den Sheets hinzuzufügen (\texttt{Commit\-Idea()}). 
	\item[Zweiter Schritt:] In der \texttt{InsertPage} ist ein Frame mit einem \texttt{TapGestureRecognizer} zu erstellen, das im dahinterliegenden Command die Navigation auf die erstellte Page ausführt.
\end{labeling}

Somit ist es dem Benutzer möglich, eine neue Idee auszuwählen und dem Brainsheet hinzuzufügen. 

\subsubsection{Neue Idee im Sheet anzeigen}
\begin{labeling}{Zweiter Schritt:}
	\item[Erster Schritt:] Jede Idee hat ihr eigenes \texttt{DataTemplate}, das das Aussehen innerhalb des Sheets definiert. Diese sind in den Content-Page-Resources der \texttt{Brain\-storming\-Page} definiert. Hier muss ein neues Template definiert werden, das die Idee und deren Attribute wie gewünscht darstellt. 
	\item[Zweiter Schritt:] Um das erstellte Template auszuwählen, muss die Klasse \texttt{Idea\-Template\-Selector} erweitert werden. Eine weitere Überprüfung des Typs auf \texttt{Qua\-lity\-Idea} reicht dabei aus, das neue Template zurückzugeben.
\end{labeling}

Hiermit sind alle benötigten Schritte erledigt und der neue Ideentyp ist integriert. 

\subsection{Exkurs: Andere Datenbank verwenden}
Der Vollständigkeitshalber soll in diesem Kapitel auch erläutert werden, wie die momentan verwendete MongoDB Datenbank ausgewechselt werden kann, sollte dies notwendig sein.

Dies kann in zwei einfachen Schritten erreicht werden.

\begin{labeling}{Zweiter Schritt:}
	\item [Erster Schritt:] Zunächst muss jedes der fünf Datenbank-Interfaces für die neue Datenbankanbindung in einer entsprechenden Klasse (z.B. MySQL\-Finding\-Service-Klasse) implementiert werden. 
	\item [Zweiter Schritt:] In einem zweiten Schritt muss im Interface nur noch mitgeteilt werden, von welcher Klasse dieses implementiert wird.

\begin{lstlisting}[caption={Andere Datenbank verwenden}, label=changeDatabase]
@ImplementedBy(MySQLFindingService.class)
public interface DBFindingInterface {...}
\end{lstlisting}
\end{labeling}
