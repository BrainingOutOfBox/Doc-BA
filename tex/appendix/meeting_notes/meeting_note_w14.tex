\hypertarget{meeting-notes-210519}{%
\section*{Meeting notes 21.05.19}\label{meeting-notes-210519}}

\hypertarget{anwesende}{%
\subsection*{Anwesende}\label{anwesende}}

\begin{itemize}
 
\item Prof. Dr. Olaf Zimmermann
\item Elias Brunner
\item Oliver Dias
\end{itemize}

\hypertarget{agenda}{%
\subsection*{Agenda}\label{agenda}}

\hypertarget{generelle-punkte}{%
\subsubsection*{Generelle Punkte}\label{generelle-punkte}}

\begin{itemize}
 
\item public Link im TestFlight für Ausstellung am 14.06
\item Inhalt des Plakates? Experten oder "Familienangehörige" als Zielgruppe? 
\item \textbf{Feedback}:
\item Zielgruppe ist vor allem Experten und Mitstudenten
\end{itemize}

\hypertarget{status-projekt}{%
\subsubsection*{Status Projekt}\label{status-projekt}}

\begin{itemize}
\item Styling 
\item \textbf{Feedback}:
\item Farben + Formen ansprechend
\item Team verlassen Plattform-Unterschiede 
\item \textbf{Feedback}:
\item Fokus auf UI und Userführung setzen
\item Testing verkürzt dafür Transition verlängert
\item \textbf{Feedback}:
\item Geht in Ordnung
\item 1x zu Dritt ein BrainstormingFinding durchführen
\item \textbf{Feedback}:
\item Register confusing wegen zwei Buttons -\textgreater{} Text auf "Create a new Account" ändern
\item Navigate back to Teamlist etwas verwirrend
\item till ersetzen durch until
\item Nicht-Moderatoren können kein Finding starten (-\textgreater{} Moderator auf Finding Ebene?)
\item Vorschlag für Anzahl Ideen (Placeholder: \# Ideas, e.g. 3)
\item Meldung "Please don't use any of the prohibited characters" zu generisch
\item Number of Ideas im Fehlertext ändern (between 0 and 100)
\item Take part 'in' Waiting State
\item Scroll to commited Idea
\item relativ stabil
\item '+'(More Ideas) und 'commit' (Add to sheet) ersetzen
\item Nach commit Toast Message mit 'Added to sheet, don't forget to scroll to see it'
\item Current Round und Current Sheet anzeigen
\item \textbf{Wichtigste Punkte aus User Test:}
\item Orientierung im Sheet schwerfällig, durch genannte UI Enhancements Abhilfe schaffen
\item Orientierung bei der Ideeneingabe ebenfalls unklar, allfälliger Navigationsbutton ("Last Idea") oder scrollen zur Ideeneingabe würde Orientierung verbessern
\item Falls nicht mehr umsetzbar dokumentieren 
\end{itemize}

\hypertarget{next-meeting}{%
\subsection*{Next Meeting}\label{next-meeting}}

28.05. 09.00 Uhr

\hypertarget{tasks-until-next-meeting}{%
\subsection*{Tasks until next meeting}\label{tasks-until-next-meeting}}

\hypertarget{prof-dr-olaf-zimmermann}{%
\subsubsection*{Prof. Dr. Olaf
Zimmermann}\label{prof-dr-olaf-zimmermann}}

\hypertarget{elias--oli}{%
\subsubsection*{Elias \& Oli}\label{elias--oli}}