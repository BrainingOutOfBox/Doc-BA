\hypertarget{meeting-notes-280219}{%
\section*{Meeting notes 28.02.19}\label{meeting-notes-280219}}

\hypertarget{anwesende}{%
\subsection*{Anwesende}\label{anwesende}}

\begin{itemize}

\item
  Prof. Dr. Olaf Zimmermann
\item
  Elias Brunner
\item
  Oliver Dias
\end{itemize}

\hypertarget{agenda}{%
\subsection*{Agenda}\label{agenda}}

\hypertarget{generelle-punkte}{%
\subsubsection*{Generelle Punkte}\label{generelle-punkte}}

\begin{itemize}

\item
  Code Review durch Silvan hilfreich
\item
  Aufgabenstellung \& Legal unterschreiben
\item
  iOS Entwickler Lizenz der HSR? 
\item \emph{\textbf{Feedback}}
\item
  Thema Lizenzen konnte nicht abgeklärt werden
\end{itemize}

\hypertarget{administratives}{%
\subsubsection*{Administratives}\label{administratives}}

\begin{itemize}

\item
  Vorankündigung: Im März / April ist O.Zimmermann 2 Wochen abwesend,
  wird 1 mal ausfallen und 1 mal Silvan als Vertretung. In der Woche vom
  25.März + Woche vom 1. April
\item
  Tag des Meetings wird etwas springen, nächste Woche Donnerstag 7.3.
  09.00 Uhr
\item
  Am 21.5 findet Meeting am Dienstag statt (O.Dias am Donnerstag /
  Freitag abwesend)
\end{itemize}

\hypertarget{review-doku}{%
\subsubsection*{Review Doku}\label{review-doku}}

\begin{itemize}
\item
  Kapitel 2.1 - 2.3 reviewen -\textgreater{} gibt es weitere funktionale
  Anforderungen?
\item
  Überarbeiteter Zeitplan (In Kapitel A5 S.31) 
\item \emph{\textbf{Feedback}}
\item
  Kapitel 2.2: Konjunktiv rausnehmen, zu vorsichtig
\item
  Arc42 ist nicht nur Doku Template, sondern zeigt auch Aktivitäten /
  Phasen \& Unterphasen, vgl. Buch Effektive SW-Architekturen.
\item
  Analysis, Synthesis, Evaluation; Solution Strategy Unterphase von
  Synthesis
\item
  Starke: Kapitel \textbf{3}, insb. 3.7 Vorgehen, Lösungsvorschläge
  entwickeln
\item
  Konkretes Beispiel geben für verständlichere Formulierung
\item
  Fehlt kreativer Teil, erwartet von Benutzer (trotzdem in Tabelle
  aufnehmen)
\item
  Es fehlt "Lösungen von bestehendem Projekt übernehmen"
\item
  Attribute Driven Design reinbringen (Für jedes QA eine Brainstorming
  Session)
\item
  Export wichtig -\textgreater{} Requirement, als MADR-Format (siehe auf
  GitHub,Markdown Template)
\item
  Velocity Apache Framework für Templating
\item
  Kapitel 2.3: Überlappung mit (womit?)
\item
  Insert zu allgemein, technischeres Verb verwenden (record, draw etc.)
\item
  UC Create MADR Export fehlt
\item
  Fully Dressed: SE Modul genauer (Bachelor Studium HSR)
\item
  Link einfügen: Prüfung auf Funktionalität? Titel der Page? Genauer in
  UC
\item
  Picture: Option für Auflösung; klein, mittel, gross
\item
  Konkrete Idee auflisten: Agile Modelling Diagram einführen (Buch +
  Website): Während Brainstorming zeichnet jmd an Whiteboard und macht
  Bild davon
\item
  Etwas tiefer in Domäne gehen
\item
  Lizenz für Fingerdrawing abklären
\item
  Pattern zu grob, ebenfalls Beispiel einfügen
\item
  Bold face solution: Frage von Problem und Name des Patterns
\item
  Evtl 2 Video Types: Reference Video (für spätere Referenz) und
  Snapshot Video (sehr kurz)
\item
  Quelle zu Landing Zones angeben
\item
  2.4: Finding zu Session ändern?
\item
  Code Review: Lakeside Mutual als Referenzprojekt anschauen
\end{itemize}

\hypertarget{file-uploaddownload}{%
\subsubsection*{File upload/download}\label{file-uploaddownload}}

\begin{itemize}

\item
  Gib es hier eine State-of-the-art Lösung? 
\item \emph{\textbf{Feedback}}
\item
  Bei grossen Files Multipart MIME Type
\end{itemize}

\hypertarget{plan}{%
\subsubsection*{Plan}\label{plan}}

\begin{itemize}

\item
  Prio 1: Anforderungen, Domainmodell anpassen, evtl. schon Mockups zeichnen
\item
  Prio 2: Prototypen für Skizzenfunktion erstellen
\item \emph{\textbf{Feedback}}
\item
  Umfang Anforderungen i.O.
\item
  Falls Meilensteine in Gefahr melden \& reagieren
\end{itemize}

\hypertarget{tasks-until-next-meeting}{%
\subsection*{Tasks until next meeting}\label{tasks-until-next-meeting}}

\hypertarget{prof-dr-olaf-zimmermann}{%
\subsubsection*{Prof. Dr. Olaf Zimmermann}\label{prof-dr-olaf-zimmermann}}

\begin{itemize}

\item
  Schickt Aufgabenstellung heute nochmals bis 11.00 Uhr
\item
  iOS Entwickler Lizenz der HSR
\end{itemize}

\hypertarget{elias--oli}{%
\subsubsection*{Elias \& Oli}\label{elias--oli}}

\begin{itemize}

\item
  Aufgabenstellung Termine in Tabelle ausfüllen
\end{itemize}