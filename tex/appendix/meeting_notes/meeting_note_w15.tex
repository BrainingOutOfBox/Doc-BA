\hypertarget{meeting-notes-280519}{%
\section*{Meeting notes 28.05.19}\label{meeting-notes-280519}}

\hypertarget{anwesende}{%
\subsection*{Anwesende}\label{anwesende}}

\begin{itemize}
\item
  Prof. Dr. Olaf Zimmermann
\item
  Elias Brunner
\item
  Oliver Dias
\end{itemize}

\hypertarget{agenda}{%
\subsection*{Agenda}\label{agenda}}

\hypertarget{generelle-punkte}{%
\subsubsection*{Generelle Punkte}\label{generelle-punkte}}

\hypertarget{status-projekt}{%
\subsubsection*{Status Projekt}\label{status-projekt}}

\begin{itemize}
\item
  Einarbeitung des Test-Feedbacks -\textgreater{} nochmalige
  Durchführung
\item
  Vergessen fürs Testen zu übertragen -\textgreater{} Durchführung
  nächste Woche
\end{itemize}

\hypertarget{ruxfcckmeldungen-dokumentation}{%
\subsubsection*{Rückmeldungen
Dokumentation}\label{ruxfcckmeldungen-dokumentation}}

\begin{itemize}
\item Abstract 
\item \emph{\textbf{Feedback}}
\item Zu lang
\item Wird verwendet um rauszufinden, ob die Arbeit für den Leser relevant ist oder nicht
\item Abschnitte 1-3 stark überarbeiten (kürzen), Abschnitte 4-5 noch etwas schärfen
\item Fluss zu sprunghaft, Abs 3 nicht mit der Frage was ist Solution Strategy beschäftigt, sondern mehr was dabei herausgekommen ist, was waren die Ergebnisse
\item Abs 5 auch ins Mgmt Summary
\item Refactoring überfordert den Leser
\item Kontext: Konzeptionelle Designentwürfe, kreative Tätigkeit, die kollaborativ erarbeitet wird
\item Für Ideenfindung Vielzahl von Methoden/Arbeitstechniken
\item Problem: Viel Papierarbeit, existiert keine App für diese Methode
\item Lösung: Erreichtes aufzeigen: wie Testergebnisse zeigen, bietet Produkt eine Lösung für das erläuterte Problem
\item Management Summary 
\item \emph{\textbf{Feedback}}
\item Besonderheiten von Solution Strategy und/oder Abwesenheit der Solution Strategy stärker ausarbeiten
\item mit Rückbezug auf Vorstudie und Fazit der Vorstudie in Mgmt Summary einbeziehen
\item Wortwahl ins Projekt integrieren
\item 'wesentlich', 'geplant' weglassen, Vorschlag: Da Software-Architekten häufig Skizzen für die Projektübersicht brauchen
\item 'komplett' mit 'umfangreich' oder 'gründlich'
\item Warum vereinfacht das Refactoring? -\textgreater{} Um die geplanten Funktionen umzusetzen
\item 1.3 Sehr gut
\item entweder mehr Domänenbezug (wieder SW-Architekten erwähnen) oder abstrakter halten (-\textgreater{} nicht zuviel versprechen!)
\item Installationsanleitung 
\item \emph{\textbf{Feedback}}
\item SBT einführen
\item Testprotokoll 
\item \emph{\textbf{Feedback}}
\item Aufrichtig und aussagekräftig, gut
\item Ergebnisbewertung 
\item \emph{\textbf{Feedback}}
\item gemäss Skala Landing Zones als gut bezeichnen (Referenz auf NFR)
\item integrieren liessen, besser Verb-Formulierung verwenden
\item Zahlen bis Zwölf schreibt man aus
\item Farben, Formen, Texte als sehr gut empfunden
\item Benutzerführung als minimal
\end{itemize}

\hypertarget{next-meeting}{%
\subsection*{Next Meeting}\label{next-meeting}}

\begin{itemize}
\item 11.6 09:00 Uhr
\end{itemize}

\hypertarget{tasks-until-next-meeting}{%
\subsection*{Tasks until next meeting}\label{tasks-until-next-meeting}}

\hypertarget{prof-dr-olaf-zimmermann}{%
\subsubsection*{Prof. Dr. Olaf
Zimmermann}\label{prof-dr-olaf-zimmermann}}

\hypertarget{elias--oli}{%
\subsubsection*{Elias \& Oli}\label{elias--oli}}