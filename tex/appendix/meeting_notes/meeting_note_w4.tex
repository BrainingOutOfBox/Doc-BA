\hypertarget{meeting-notes-120319}{%
\section*{Meeting notes 12.03.19}\label{meeting-notes-120319}}

\hypertarget{anwesende}{%
\subsection*{Anwesende}\label{anwesende}}

\begin{itemize}

\item
  Prof. Dr. Olaf Zimmermann
\item
  Elias Brunner
\item
  Oliver Dias
\end{itemize}

\hypertarget{agenda}{%
\subsection*{Agenda}\label{agenda}}

\hypertarget{generelle-punkte}{%
\subsubsection*{Generelle Punkte}\label{generelle-punkte}}

\begin{itemize}
\item
  Deployment mit AppCenter -\textgreater{} Meeting mit Toni
\item
  \emph{\textbf{Feedback}}: OK, versuchen Aufwand schlank zu halten
\item
  In wie weit ist es sinnvoll, Themen wie Deployment oder Ablaufdiagramm
  nochmals in den techn. Bericht einfliessen zu lassen?
\item
  \emph{\textbf{Feedback}} Reinnehmen, mit Referenz auf SA
\item
  Zeitplan update
\item
  Zeitplan wurde gezeigt und wurde akzeptiert
\end{itemize}

\hypertarget{review-doku}{%
\subsubsection*{Review Doku}\label{review-doku}}

\begin{itemize}

\item
  \emph{\textbf{Feedback}} Kapitel 2.6.1
\item
  Online Review
\item
  Nochmals auf editorielle Korrektheit durchgehen
\end{itemize}

\hypertarget{prototype}{%
\subsubsection*{Prototype}\label{prototype}}

\begin{itemize}

\item
  Prototyp wurde gezeigt und akzeptiert
\end{itemize}

\hypertarget{mockups}{%
\subsubsection*{Mockups}\label{mockups}}

\begin{itemize}

\item
  Mockup wurde gezeigt
\item
  \emph{\textbf{Feedback}} Mögliche Ideas nach Wichtigkeit sortieren
\item
  Bei Patterns: Bold Face Solution (gemäss MAP) wäre anstelle von "some
  description" in Mockup.
\end{itemize}

\hypertarget{terminfindung}{%
\subsubsection*{Terminfindung}\label{terminfindung}}

\begin{itemize}
\item
  Experte und Gegenleser sind bekannt
\item
  Termin Präsentation Orientierung: 30.04 Vormittag
\item
  Ausweichtermin: 02.05 09.00-10.00 Uhr
\item
  Ausweichtermin 2: 16.04 Vormittag
\item
  Meeting nächste Woche: 21.03 09.00 Uhr
\end{itemize}

\hypertarget{anregung}{%
\subsubsection*{Anregung}\label{anregung}}

\begin{itemize}

\item
  World Café als kreative Brainstorming Methode in der Gruppe
\end{itemize}

\hypertarget{tasks-until-next-meeting}{%
\subsection*{Tasks until next meeting}\label{tasks-until-next-meeting}}

\hypertarget{prof-dr-olaf-zimmermann}{%
\subsubsection*{Prof. Dr. Olaf
Zimmermann}\label{prof-dr-olaf-zimmermann}}

\begin{itemize}

\item
  Bescheid wegen Termin mit Experte / evtl. Gegenleser
\end{itemize}

\hypertarget{elias--oli}{%
\subsubsection*{Elias \& Oli}\label{elias--oli}}