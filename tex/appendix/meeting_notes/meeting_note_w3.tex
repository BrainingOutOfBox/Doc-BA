\hypertarget{meeting-notes-070319}{%
\section*{Meeting notes 07.03.19}\label{meeting-notes-070319}}

\hypertarget{anwesende}{%
\subsection*{Anwesende}\label{anwesende}}

\begin{itemize}

\item
  Prof. Dr. Olaf Zimmermann
\item
  Elias Brunner
\item
  Oliver Dias
\end{itemize}

\hypertarget{agenda}{%
\subsection*{Agenda}\label{agenda}}

\hypertarget{generelle-punkte}{%
\subsubsection*{Generelle Punkte}\label{generelle-punkte}}

\begin{itemize}

\item
  iOS Entwickler Lizenz der HSR 
\item \emph{\textbf{Feedback}} Toni Sutter kann hier weiter helfen.
\item
  Referenz SA 
\item \emph{\textbf{Feedback}} Dauert noch einen Moment. Unsere SA wird aber noch veröffentlicht.
\item
  Legal 
\item \emph{\textbf{Feedback}} unterschriebene Aufgabenstellung, Einverständniskeitserklärung etc. wurde  an Betreuer zurück gegeben.
\item
  Punkte von Betreuer: 
\item \emph{\textbf{Feedback}}
\item
  Nächstes Meeting findet am 12.03 16:00Uhr statt.
\item
  Rückmeldung zum Protokoll: Besser ausformulieren (keine
  Umgangssprache); nur auf Wichtigstes/Entscheide fokussieren.
\item
  Für externe Ressourcen (Meeting mit Mitarbeiter, etc. \textgreater{}
  1h) muss Betreuer informiert werden.
\end{itemize}

\hypertarget{review-doku}{%
\subsubsection*{Review Doku}\label{review-doku}}

\begin{itemize}
\item
  \emph{\textbf{Feedback}} Kapitel 2.2.2
\item
  generell i.O. allerdings auf Wortwahl achten
\item
  \emph{\textbf{Feedback}} Kapitel 2.2.3
\item
  generell i.O. allerdings auf Wortwahl achten
\item
  \emph{\textbf{Feedback}} Kapitel 2.4
\item
  Diskussion Domain Modell -\textgreater{} separate Findingtypen?
\item
  Inbezugnahme auf Erweiterbarkeit in Aufgabenstellung
\item
  Wir präsentieren Herrn Zimmermann die Idee/Vorteile einer einfachen Struktur ohne das Konzept der verschiedenen Finding-Typen. Nach unserer Meinung kann so die Erweiterbarkeit für zukünftige Arbeiten an der App einfacher umgesetzt werden. Zu beachten gilt dabei vorallem die Usability für den Endnutzer. Die Navigation muss einfach bleiben, um sich schnell für eine Ideenart zu entscheiden.
\item
  Besonderes Augenmerk gilt daher folgenden Punkten:
  \begin{itemize}
  \item
    Usability (NFR) -\textgreater{} Schnelligkeit bei der Navigation
  \item Aufwand für andere Entwickler bei Erweiterung (NFR)
  \end{itemize}
\end{itemize}

\hypertarget{next-meeting}{%
\subsection*{Next Meeting}\label{next-meeting}}

\begin{itemize}

\item
  12.03 16:00Uhr
\item
  Die Traktanden können auch noch am Montag geschickt werden.
\end{itemize}

\hypertarget{tasks-until-next-meeting}{%
\subsection*{Tasks until next meeting}\label{tasks-until-next-meeting}}

\hypertarget{prof-dr-olaf-zimmermann}{%
\subsubsection*{Prof. Dr. Olaf
Zimmermann}\label{prof-dr-olaf-zimmermann}}

\hypertarget{elias--oli}{%
\subsubsection*{Elias \& Oli}\label{elias--oli}}