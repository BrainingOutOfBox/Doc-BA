\hypertarget{meeting-notes-040419}{%
\section*{Meeting notes 04.04.19}\label{meeting-notes-040419}}

\hypertarget{anwesende}{%
\subsection*{Anwesende}\label{anwesende}}

\begin{itemize}

\item
  Prof. Dr. Olaf Zimmermann
\item
  Elias Brunner
\item
  Oliver Dias
\end{itemize}

\hypertarget{agenda}{%
\subsection*{Agenda}\label{agenda}}

\hypertarget{generelle-punkte}{%
\subsubsection*{Generelle Punkte}\label{generelle-punkte}}

\begin{itemize}
\item Vorschlag von uns: Meetings in nächsten Wochen nur bei Bedarf
\item \emph{\textbf{Feedback}} Akzeptiert
\end{itemize}

\hypertarget{status-der-arbeiten}{%
\subsubsection*{Status der Arbeiten}\label{status-der-arbeiten}}
\begin{itemize}
\item Status Refactoring 
\item \emph{Status wurde mündlich erläutert und aufgenommen}
\item Prototyp für Export-Funktion 
\item \emph{Prototyp wurde gezeigt und akzeptiert}
\item Prototyp für weitere Ideen-Typen 
\item \emph{Prototyp wurde gezeigt und akzeptiert}
\end{itemize}

\hypertarget{feedback-dokumentation}{%
\subsubsection*{Feedback Dokumentation}\label{feedback-dokumentation}}

\begin{itemize}
\item Feedback Ausblick Kapitel 2.9.3 
\item \emph{\textbf{Feedback}}:
\item Zu umgangssprachlich
\item Tipps zu Kapitel Ausblick wurden gegeben
\item Feedback Sequenzdiagramm 2.3.1.3
\item \emph{\textbf{Feedback}}:
\item Sequenzdiagramm grösser (querformat)
\end{itemize}

\hypertarget{next-meetings}{%
\subsubsection*{Next Meetings}\label{next-meetings}}

\begin{itemize}

\item
  11.4. 09.00 Uhr provisorisch, bei Absage spätestens Vorabend bis 17.00 Uhr Bescheid mit Statusupdate
\item
  16.4. 10.00 Uhr fix
\end{itemize}

\hypertarget{tasks-until-next-meeting}{%
\subsection*{Tasks until next meeting}\label{tasks-until-next-meeting}}

\hypertarget{prof-dr-olaf-zimmermann}{%
\subsubsection*{Prof. Dr. Olaf
Zimmermann}\label{prof-dr-olaf-zimmermann}}

\hypertarget{elias--oli}{%
\subsubsection*{Elias \& Oli}\label{elias--oli}}

\begin{itemize}
\item Mail mit Prism Framework Version fürs Fragen nach Experten
\end{itemize}