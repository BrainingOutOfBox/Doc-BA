\hypertarget{meeting-notes-220219}{%
\section*{Meeting notes 22.02.19}\label{meeting-notes-220219}}

\hypertarget{anwesende}{%
\subsection*{Anwesende}\label{anwesende}}

\begin{itemize}

\item
  Prof. Dr. Olaf Zimmermann
\item
  Elias Brunner
\item
  Oliver Dias
\end{itemize}

\hypertarget{agenda}{%
\subsection*{Agenda}\label{agenda}}

\hypertarget{generelle-punkte}{%
\subsubsection*{Generelle Punkte}\label{generelle-punkte}}

\begin{itemize}
\item
  Öffentliches GitHub Repo für Code / Doku?
\item \emph{\textbf{Entscheid}}
\item
  Herr Zimmermann empfielt zu privaten Repos wegen Crawlers etc.
\item
  Formal gibt es keine Richtlinien
\item
  Meinungen/Aussagen von Herrn Zimmermann werden nicht auf GitHub
  gestellt
\item
  Wird so weitergemacht
\item
  Projektplan vorstellen 
\item \emph{\textbf{Entscheid}}
\item
  etwas grob aber auf einer hohen Ebene gesehen ok
\item
  Analyse vlt. etwas zu knapp
\item
  Vorschlag von Herrn Zimmermann: Code Review zur SA von einen MA IFS
\item
  Upload in den AppStore / PlayStore 
\item \emph{\textbf{Entscheid}}
\item
  Lieber eine Plattform als keine Plattform -\textgreater{}
  wahrscheinlich PlayStore
\item
  DevAccount fürs AppCenter, Azure Subscription
\item
  Möglichkeit für iOS Dev License? Android License?
\item \emph{\textbf{Entscheid}}
\item
  Schauen, ob es Unterstützung von Seiten HSR gibt (iOS Dev / Android
  License)
\item
  Azure Subscription kann so beibehalten werden
\item
  Präsentation so früh wie möglich?
\item
  Anforderungen Präsentation
\item
  Wer ist unser Experte? 
\item \emph{\textbf{Entscheid}}
\item
  Für Herrn Zimmermann auch gerne möglich
\item
  Hängt aber von externen Stakeholdern ab
\item
  Kann ab Ostern bekannt gegeben werden
\item
  Für die Zwischenpräsentation kann auch ein Draft abgegeben werden
\end{itemize}

\hypertarget{review-aufgabenstellung}{%
\subsubsection*{Review Aufgabenstellung}\label{review-aufgabenstellung}}

\begin{itemize}

\item
  gegenseitiges Einverständnis (Solution Strategy; inwieweit anbieten?)
\item
  Schreibfehler 'zu' 
\item \emph{\textbf{Entscheid}}
\item
  Wie die Solution Strategy einzubringen ist, ist auch noch nicht ganz
  sicher
\item
  Aus der Analyse sollte bekannt sein, was die Solution Strategy mit der
  Methode 635 gemeinsam/anders hat -\textgreater{} tabellarisch
\item
  Gibt es Begriffe aus beiden Domänen, welche aber dasselbe bedeuten
\item
  Bei den (API) Patterns könnten die Foundation Patterns genommen werden
  -\textgreater{} als Link auf die HP
  \href{https://microservice-api-patterns.org}{https://microservice-api-patterns.org}?
\end{itemize}

\hypertarget{feedback-zur-studienarbeit}{%
\subsubsection*{Feedback zur
Studienarbeit}\label{feedback-zur-studienarbeit}}

\begin{itemize}
\item
  Für die BA könnten mehr wissenschaftliche Arbeiten (Buch, Google
  Scholar) genommen werden
\item
  Protokolle wurden immer knapper (für BA zu verbessern)
\item
  Es wurde erkannt, dass Anregungen aus Meetings etc. umgesetzt wurden
  -\textgreater{} so beibehalten
\item
  Sprache (bis auf ein paar umgangssprachlige Aussagen) und
  Dokumentation war gut
\item
  klares Verbesserungspotenzial liegt im Code
\item
  Es wirkt nicht objektorientiert (flüchtiger, erster Eindruck)
  -\textgreater{} prüfen, ob stärker objektorientiert programmiert
  werden kann
\item
  BrainwaveFinding muss auch Logik für Timer beinhalten
\item
  mehr Layer (Service-Layer für DB Zugriff)
\item
  Konfigurationseigenschaft sind hardcodiert
\end{itemize}

\hypertarget{tasks-until-next-meeting}{%
\subsection*{Tasks until next meeting}\label{tasks-until-next-meeting}}

\hypertarget{prof-dr-olaf-zimmermann}{%
\subsubsection*{Prof. Dr. Olaf
Zimmermann}\label{prof-dr-olaf-zimmermann}}

\begin{itemize}
\item
  Aufgabenstellung v1.0 unterschreiben
\item
  Code Review durch IFS Mitarbeiter klären
\item
  Klären, ob die Dev License der HSR verwendet werden darf
  (Team-Kollaboration)
\end{itemize}

\hypertarget{elias--oli}{%
\subsubsection*{Elias \& Oli}\label{elias--oli}}

\begin{itemize}
\item
  nächstes Meeting wäre nächsten Donnerstag 28.02 09:00Uhr
\end{itemize}