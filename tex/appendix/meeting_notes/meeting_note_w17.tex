\hypertarget{meeting-notes-110619}{%
\section*{Meeting notes 11.06.19}\label{meeting-notes-110619}}

\hypertarget{anwesende}{%
\subsection*{Anwesende}\label{anwesende}}

\begin{itemize}
\item
  Prof. Dr. Olaf Zimmermann
\item
  Elias Brunner
\item
  Oliver Dias
\end{itemize}

\hypertarget{agenda}{%
\subsection*{Agenda}\label{agenda}}

\hypertarget{ruxfcckmeldungen-dokumentation}{%
\subsubsection*{Rückmeldungen
Dokumentation}\label{ruxfcckmeldungen-dokumentation}}

\begin{itemize}
\item
  Herausforderung 
\item \emph{\textbf{Feedback}}
\item
  andere Position? (nur ein Titel)
\item
  Implementation Frontend 
\item \emph{\textbf{Feedback}}
\item
  Inhalt, Länge, Menge gut
\item
  fetter Text? "Wichtig zu erwähnen" konsistent halten
\item
  Soll/Ist 
\item \emph{\textbf{Feedback}}
\item
  Diskussion wurde geklärt
\item
  Zeitauswertung 
\item \emph{\textbf{Feedback}}
\item
  Schrift der Legende grösser
\item
  evtl. Interpretation der aufgewändeten Zeiten
\item
  Ideenerweiterung 
\item \emph{\textbf{Feedback}}
\item
  Informationsmenge und Formales sieht gut aus
\end{itemize}

\hypertarget{fragen-dokumentation}{%
\subsubsection*{Fragen Dokumentation}\label{fragen-dokumentation}}

\begin{itemize}
\item
  Klärung Deployment Units in Architekturdokumentation
\item \emph{\textbf{Feedback}}
\item
  Diskussion wurde geklärt
\end{itemize}

\hypertarget{fragen-zur-schlusspruxe4sentation}{%
\subsubsection*{Fragen zur
Schlusspräsentation}\label{fragen-zur-schlusspruxe4sentation}}

\begin{itemize}
\item
  Was ist der Inhalt / Unterschied zur Zwischenpräsentation ?
\item \emph{\textbf{Feedback}}
\item
  Ziel: Zeitmanagement, Ausgewogenheit (Folien, Demo), Resultat,
\item
  sicher im Inhalt sein (alle kennen alles)
\item
  20min (10min/10min) inkl. Demo + 5-10min Q/A
\end{itemize}

\hypertarget{status-projekt}{%
\subsubsection*{Status Projekt}\label{status-projekt}}

\begin{itemize}
\item
  Einarbeitung des Test-Feedbacks -\textgreater{} nochmalige
  Durchführung
\item
  Abstruz auf iPhone und iPad von Betreuer
\item
  Userführung hat sich verbessert.
\end{itemize}

\hypertarget{next-meeting}{%
\subsection*{Next Meeting}\label{next-meeting}}

keines

\hypertarget{tasks-until-next-meeting}{%
\subsection*{Tasks until next meeting}\label{tasks-until-next-meeting}}

\hypertarget{prof-dr-olaf-zimmermann}{%
\subsubsection*{Prof. Dr. Olaf
Zimmermann}\label{prof-dr-olaf-zimmermann}}

\hypertarget{elias--oli}{%
\subsubsection*{Elias \& Oli}\label{elias--oli}}